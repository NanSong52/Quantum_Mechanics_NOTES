\documentclass[a4paper, 12pt]{article}
\usepackage{matnoble-doc-en}
\usepackage{amsmath}
\usepackage{simpler-wick}
\newcommand\Item[1][]{%
  \ifx\relax#1\relax  \item \else \item[#1] \fi
  \abovedisplayskip=0pt\abovedisplayshortskip=0pt~\vspace*{-\baselineskip}}

\begin{document}

\title{\bf {CHEM400/740: Quantum Mechanics in Chemistry\\ Chapter\#12: Configuration Interaction(CI) and Coupled Cluster(CC)}} \author{\bf
  \href{http://scienide2.uwaterloo.ca/~nooijen/website_new_20_10_2011/About.html}{Marcel Nooijen}} \date{}
  
\pagestyle{fancy} \fancyhead[L]{\textcolor{PrimaryColor}{CHEM400/740: Quantum Mechanics in Chemistry}} \fancyhead[R]{\textcolor{PrimaryColor}{2021 Winter}}


\maketitle
\tableofcontents

\clearpage


\section{Overview} 
Up to now we considered the use of small expansions, `Configuration interaction' to calculate approximate.
\begin{itemize}
	\item [-] Ionization potentials
	\item [-] Electron affinities
	\item [-] Excitation energies
\end{itemize}
\tab Using the difference of variational result and Hartree-Fock ground state energy. $E_{HF} = \langle HF|\hat{H}|HF\rangle $ is the lowest possible energy for a wave function that consists of a single determinant: populate an optimal set of molecular orbitals. This gives good total energies (at equilibrium geometry typically 99.9\% of the total energy.)\\
\tab 1\% of $\sim$ 100 Hartree = 0.1 Hartree $\approx$ 65 kcal/mol $\sim$ 250 kJ/mol \\
\tab On a chemical scale these are huge errors. We cannot use HF to calculate reaction energies. They are quite wrong. Geometries and frequencies are typically more reasonable.\\
\tab Bottom line: we need to go beyond $|HF\rangle$. The simplest approach is to use the variational principle once again. 

In what follows I will discuss briefly various approaches to the electronic structure problem. Let me give a brief outline first: 
\begin{itemize}
	\item [A.] The basic approach: single reference configuration interaction (CI), CISD, CISDT: \\
	- Not size-consistent: $E_{AB}\neq E_A+E_B$ for non-interacting system $A, B $.\\
	- Does not describe bond breaking.
	\item [B.] Fixing the size-consistency problem: Coupled Cluster with $|\psi\rangle = e^{\hat{T}}|HF\rangle$\\
	-CCSD(T), excellent for good HF molecules.\\
	-Very poor convergence with size of one-particle basis set.
	\item [C.] Solving the multi-reference problem (bond breaking):\\
	-MR-CI \\
	-MR-CI(+Q) ($\sim$ size-consistent)\\
	-MR-CC 
	\item [D.] Methods for excited states: \\
	- Equation of Coupled Cluster (also SAC-CI, CCLRT). \\
	- Similarity transformed EOMCC.
	\item [E.] Wave function methods for large molecules: \\
	- Local correlation.\\
	- Explicit $r_{12}$ methods ($F_{12}$ method). Solve the basis set convergence problem.
\end{itemize}

I anticipate that the remainder of the course will be discussing these items. My own research concerns developing new method in electronic structure theory, that can address various issues. I will give examples from this research environment. 




\section{Configuration Interaction (CI)}
\subsection{Configuration Interaction}
Conceptually the simplest approach would be to make the expansion: 
\begin{IEEEeqnarray}{rLl}
|\psi \rangle &= C_o|HF\rangle + \sum_{r,a} C_a^r \hat{r}^\dagger \hat{a} |HF\rangle + \sum_{\substack{r<s \\ a<b}}C_{ab}^{rs} \hat{r}^\dagger a s^\dagger b |HF\rangle + \cdots \notag \\
&= \sum_\lambda C_\lambda |\phi_\lambda \rangle
 \end{IEEEeqnarray}
\tab Note that $a,b,c,d$ are occupied orbitals, $p,q,r,s$ are virtual orbitals in $|HF\rangle$. \\
\tab \tab $NAO$ = \# of occ + \# of vir 

In CISD we limit the expansion to go up to double substitutions. \\
\tab In CISDT we limit the expansion to go up to triple substitutions. \\
\tab Full CI: allow up to N-fold substitutions for N-electrons. 

Using techniques we discussed, you could evaluate the matrix elements from zero, singles, and doubles. Example:
\begin{IEEEeqnarray}{rLl}
H_{SO} &= \langle HF|\hat{b}^\dagger \hat{s}\hat{H}|HF\rangle \notag \\
&= h_{bs}+ \sum_c \langle bc||sc\rangle \\
&= f_{bs}
 \end{IEEEeqnarray}
\tab For Hartree-Fock single, this has a special meaning, as HF orbitals can be shown to satisfy $f_{bs}=0 $ for all $b,s$. This is called the Brillouin coordinate. It implies that the energy of the Hartree Fock wave function cannot be improved by mixing in single excitations only. 
\begin{table}[!htbp]
\centering
\begin{tabular}{|c|c|c|}
\hline 
$\hat{H}$ & $HF$ & $S$ \\
\hline  
$HF$ & $E_{HF}$ & 0  \\
\hline 
$S$ & 0 & $H_{SS}$ \\
\hline 
\end{tabular}
\end{table}\\
\tab This is equivalent to the fact that the occupied orbitals are optimal in HF (as an aside).\\


 Other types of matrix-elements: 
\begin{IEEEeqnarray}{rLl}
H_{OO} &= \langle HF| a^\dagger b^\dagger s r \hat{H}|HF\rangle \notag \\
&= \langle ab||sr \rangle \\
 H_{SD} &= \langle HF|( c^\dagger t) \hat{H} (s^\dagger r^\dagger ab )|HF\rangle 
 \end{IEEEeqnarray}
\tab All these matrix-elements can be evaluated using the `line' algorithm we discusses. It does get tedious but no new principles are involved. \\\
\tab If we have all matrix elements evaluated we can in principle diagonalize the resulting Hamiltonian. This is not how it is done in practice. This matrix is too large to store(even on disk). 




\subsection{Apply in Practice}
 Let us discuss what is done in practice (more or less):
\begin{itemize}
	\item [i.] Intermediate Normalization:
\begin{IEEEeqnarray}{rLl}
 |\psi\rangle  &= C_o|HF\rangle +\sum_\lambda C_\lambda |\phi_\lambda\rangle \\
\hat{H}|\psi\rangle &= E|\psi\rangle
 \end{IEEEeqnarray}
	\tab \tab 	\tab \tab 	 (for finite set of configurations)\\
	 I can always multiply $|\psi\rangle$ by a constant, and it is still an eigenvector. 
	\begin{IEEEeqnarray}{rLl}
\hat{H}| \psi\rangle \lambda = E|\psi\rangle \lambda 
 \end{IEEEeqnarray} 
	 Let us choose this constant to be $ \lambda = 1/C_0$, $C_\lambda = C_\lambda / C_0$\\
	 Then the wave function can be written: 
	\begin{IEEEeqnarray}{rLl}
|\psi\rangle = |\phi_{HF}\rangle +\sum_\lambda C_\lambda|\phi_\lambda \rangle
 \end{IEEEeqnarray} 	 	 
	 Normalization: $\langle \phi_{HF} |\psi\rangle =1$
	 \begin{IEEEeqnarray}{rLl}
\langle \phi_{HF}|\psi\rangle &= \langle \phi_{HF} \Big(|\phi_{HF}\rangle +\sum_\lambda C_\lambda|\phi_\lambda  \rangle \Big) \notag \\
&= 1+ \sum_{\lambda \neq 0}\langle \phi_{HF}|\phi_\lambda\rangle  \notag \\
& =1
 \end{IEEEeqnarray} 
	 This  	is called intermediate normalization.
		\begin{IEEEeqnarray}{rLl}
|\psi\rangle = (1+\hat{C})|\phi_{HF} \rangle
 \end{IEEEeqnarray} 
	This also leads to a convenient equation for energy: 
\begin{IEEEeqnarray}{rLl}
\langle \phi_{HF}|\hat{H}|\psi\rangle = \langle \phi_{HF}| E |\psi\rangle = E\langle \phi_{HF}|\psi\rangle =E
 \end{IEEEeqnarray} 
	Other equations for single excitation and double excitation: 
		\begin{IEEEeqnarray}{rLl}
\langle \phi_a^r \ | (\hat{H}-E)(1-\hat{C})|HF\rangle &=0 \\
\langle \phi_{ab}^{rr} | (\hat{H}-E)(1-\hat{C})|HF\rangle &=0
 \end{IEEEeqnarray} 
 	
	\item [ii.] Correlation Energy:
\\	In practice these equations are solved iteratively.\\
 If we have a trial operator (guess operator) $ \tilde{C}, \tilde{E}$.	
	\begin{IEEEeqnarray}{rLl}
\tilde{E} &= \langle \phi_{HF}|H(1+\tilde{C})|HF\rangle\notag \\
&= E_{HF} + \langle \phi_{HF}|H\tilde{C}|HF\rangle \notag \\
&= E_{HF} +\tilde{E}_c \\
E-E_{HF} &= \langle \phi_{HF}|\hat{H} \hat{C}|\phi_{HF}\rangle =E_c
 \end{IEEEeqnarray}
 \tab $E_c$ is so-called correlation energy, the piece extra beyond Hartree Fock due to the mixing in of additional determinants in the wave function. Since the operator $\tilde{C}$ is not quite correct, and can calculate the error (residual), $R_\lambda$. 
 	\begin{IEEEeqnarray}{rLl}
 R_\lambda =     \left\{ \begin{array}{rcl}
\langle \phi_\lambda^r |(\hat{E}-\tilde{E} ) (1+\tilde{C})|HF\rangle \\
\langle \phi_{ac}^{rc} |(\hat{E}-\tilde{E} ) (1+\tilde{C})|HF\rangle 
                \end{array}\right.
 \end{IEEEeqnarray}
 This essentially require the calculation of $\hat{H}\hat{C}$ and this operation can be expressed using the one- and two- electron integrals and coefficients $\hat{C}_\lambda$. \\
The crux of an efficient CI calculation is the calculation of the error vector. \\
 The correction of error vector can bring the vector close to convergence.
  	\begin{IEEEeqnarray}{rLl}
\triangle C_a^r \ &= (\varepsilon_a -\varepsilon_r )^{-1} R_a^r \\
\triangle C_{ab}^{rc} &= (\varepsilon_a+\varepsilon_b-\varepsilon_r-\varepsilon_s )^{-1} R_{ab}^{rs} \\
\tilde{L}^{n+1} &= \tilde{C}^{(n)} +\triangle C^{(n)}
 \end{IEEEeqnarray}
  Various clever scheme exist to improve and accelerate convergence. \\ All of this is cheap. The expensive part is the calculation of $\hat{H}\tilde{C}$.(DIIS)\\
The most expensive contribution is $\sum_{tu}\langle rs||tu\rangle C_{ab}^{tu} \Rightarrow R_{ab}^{rs} $ which scales as $n_0^2 n_v^4$. \\
 	\tab $n_0:$ the number of occupied orbitals \\
 	\tab $n_v:$ the number of virtual orbitals\\
It is said that CI scales as $n^6$: doubling the size of the system increases the cost by a factor $2^6 = 64$. 
 	In contrast HF (or DFT) scales as $n^3$. This is the reason(primarily) why DFT is so popular.
\end{itemize}

\subsection{Size-Consistency and CI}

CISD or higher order single-reference CI is hardly ever use in practice. It is conceptually straightforward, but it suffers from a major problem: lack of size-consistency. 

Discussion: \\
\tab Let us consider a system consisting of 2 non-interaction subsystems. Orbitals are either localized on A, or on B. Integrals that contain orbitals on both sides (mixed) can be assumed 0. Then, 
  	\begin{IEEEeqnarray}{rLl}
\hat{H} &= \hat{H}_A+\hat{H}_B \\
E &=E_A+E_B \\
\psi &= \psi_A \psi_B \equiv A(\psi_A\psi_B ) \\
(\hat{H}_A+\hat{H}_B) \psi_A \psi_B &= (\hat{H}_A \psi_A )\psi_B+ \psi_A(\hat{H}_B\psi+B ) \notag \\
&= (\bar{E}_A+\bar{E}_B) \psi_A \psi_B
 \end{IEEEeqnarray}
 \tab Some care is needed for antisymmetrizer. \\
\tab For the CI equations we would want a separate equation for $\hat{C}_A$ and for $\hat{C}_B$. 
 \begin{IEEEeqnarray}{rLl}
\langle \phi_\lambda^A| (H_N^A -E_C^A)(1+\tilde{C}_A )|\phi_{HF}\rangle   &= 0 \\
\langle \phi_\lambda^B| (H_N^B -E_C^B)(1+\tilde{C}_B )|\phi_{HF}\rangle   &= 0 
 \end{IEEEeqnarray}
 \tab Here $H_N=H-E_{HF} $\\
\tab This is more or less what the CI equations reduce to, except for one big difference: instead of $E_c^a, E_c^B$ the total correlation energy in the equations. This results a very significant change, and introduce a large error. The equations do not reduce to two decoupled for systems A and B.\\



 Alternative way might think that a consistent wave function for the AB system should look like: 
 \begin{IEEEeqnarray}{rLl}
\psi_{AB} &\approx (1+ \hat{C}_A)(1+ \hat{C}_B)|HF\rangle \notag \\
&=(1+\hat{C}_A+\hat{C}_B +\hat{C}_A\hat{C}_B )|HF\rangle \\
|HF\rangle &=| \underbrace{ (a_1^\dagger \cdots a_k^\dagger )}_{ A}\underbrace{ (b_1^\dagger \cdots b_l^\dagger}_{B} )vac\rangle
 \end{IEEEeqnarray}
\tab Separate properly: $\hat{C}_A\hat{C}_B|HF\rangle \Rightarrow$ quadruple excitation. 


This size consistency problem causes troubles in general: \\
\tab In general, CI does not scale correctly with the size of the system. Correct scaling is called size-extensivity or size-consistency. The scaling in CI can be solved using a different approach: coupled cluster. \\
\tab Before we consider CC, I will address the other, independent problem in CI: the description of bond-breaking, e.g. in $F_2$.\\
\tab Around equilibrium the wave function is well described  by the single determinant $|HF\rangle$, and $(1+\hat{C})|HF\rangle$ would give reasonable description (but for extensivity issue).

As the bond in $F_2$ breaks the qualitative wave function is given by $C_1|\cdots \sigma\bar{\sigma } |+C_2|\cdots \sigma^*\bar{\sigma}^* |$, where $C_1$ and $C_2$ have equal magnitude. \\
\tab In CISD we only indicate single(S), double(D) excitations from the first $|HF\rangle$ determinant. The second determinants and excitations out of it are equal importance. These are often quadruple excitations, e.g. $(\sigma \rightarrow \sigma^* )^2 \ (\pi \rightarrow r )^2$ and these are not included in the wave function (CISD). \\
\tab Therefore the energy carries are quite in error. This problem occurs whenever the wave function needs multiple determinants for its qualitative description. \\
\tab Example: \\
\tab\tab \tab - bond-breaking\\
\tab\tab \tab - biradicals \\
\tab\tab \tab - transition metal systems \\
\tab\tab \tab - (low lying excited states)\\
\tab Also for this problem, there is a solution: \fbox{multi-reference CI}. This approach still suffer (slightly) from the size-consistency problem. \\




One additional note regarding size-consistency: \\
\tab If I do CISD in n identical non-interacting subsystem. e.g. $n \ H_2$ molecule, widely separated: $E_c(n) \ \approx \ \sqrt{n} E_c(1)$ as $n\rightarrow \infty$\\
\tab The correct behaviour would of course be $nE_c(1)$. This illustrates again the poor behaviour of truncated CI.
























\section{Coupled Cluster Theory (CC)}
\subsection{Coupled Cluster}

In Coupled Cluster theory the same type of excitation operators are used as in CI.
 \begin{IEEEeqnarray}{rLl}
\hat{T} \ &=\hat{T}_1+ \hat{T}_2+\hat{T}_3 +\cdots  \\
\hat{T}_1 &= \sum_{r,a} t_a^r \ \hat{r}^\dagger \hat{a}\\
\hat{T}_2 &=\frac{1}{4} \sum_{\substack{ r,a\\a,b}} t_{ab}^{rs} \ \hat{r}^\dagger \hat{a} \hat{s}^\dagger\hat{b} \\
\hat{T}_3 &=\frac{1}{(3!)^2} \sum_{\substack{ r,s,t\\a,b,c}} t_{abc}^{rst} \ \hat{r}^\dagger \hat{a}\hat{s}^\dagger \hat{b} \hat{t}^\dagger c
 \end{IEEEeqnarray}
\tab Note: 
\begin{itemize}
	\item $\hat{T}_1 $: singles, $\hat{T}_2$: doubles, $\hat{T}_3 $: triples. 
	\item In CCSD one includes $\hat{T}_1, \hat{T}_2$. \\
	In CCD only includes  $\hat{T}_2$ \\
	In CCSDT up to $\hat{T}_3$
\end{itemize}

The differences with CI is the parametrization of the wave function (and the values of the coefficients)\\
\tab The Coupled Cluster wave function: 
 \begin{IEEEeqnarray}{rLl}
|\psi\rangle =e^{\hat{T}}|\phi_0\rangle
 \end{IEEEeqnarray}
\tab $|\phi_0\rangle$: single determinant, typically $|HF\rangle$.\\
\tab The simplest physically meaningful case is $\hat{T} = \hat{T}_2$ (CCD). Then,
 \begin{IEEEeqnarray}{rLl}
|\psi\rangle &=e^{\hat{T}_2}|\phi_0\rangle \notag \\
&= (1+\hat{T}_2+\frac{1}{2}\hat{T}_2^2+\frac{1}{3!}\hat{T}_2^3+\cdots )|\phi_0\rangle \notag \\
&= |\phi_0+\frac{1}{4}\sum_{\substack{r,s\\a,b} }t^{rs}_{ab} \ \hat{r}^\dagger\hat{a}\hat{s}^\dagger \hat{b}|\phi_0\rangle \notag \\
&+ \frac{ 1}{2} (\frac{1}{4} )^2 \sum_{\substack{r,s\\a,b} }\sum_{\substack{t,u\\c,d} } \ \hat{r}^\dagger \hat{a}\hat{s}^\dagger \hat{b}\hat{t}^\dagger \hat{c}\hat{u}^\dagger \hat{d}|\phi_0 \rangle  \notag \\
&+ \frac{1}{3!}\frac{1}{4}^3 \sum_{\substack{r,s\\a,b} }\sum_{\substack{t,u\\c,d} } \sum_{\substack{v,w\\e,f} }  (\cdots \cdots )|\phi_0 \rangle
 \end{IEEEeqnarray}
\tab Acting with the exponential includes higher excitations: doubles, quadruples, sextuples, octuples. The coefficients of the higher excitations are determined by the elementary $t_{ab}^{rs}$ coefficients. It is fairly easy to verify that such an exponential wave function is in principle exact, just as in FCI. The identification of $\hat{C}$ and $\hat{T}$ is as follows:
 \begin{IEEEeqnarray}{rLl}
\hat{C}_1 = \hat{T}_1 \tab &\tab \hat{T}_1 = \hat{C}_1 \\
\hat{C}_2 = \hat{T}_2+\frac{1}{2}\hat{T}_1^2 \tab&\tab \hat{T}_2=\hat{C}_2-\frac{1}{2}\hat{T}_1^2 \\
\hat{C}_3 = \hat{T}_3+\hat{T}_1\hat{T}_2+\frac{1}{3!}\hat{T}_1^3 \tab &\tab \hat{T}_3=\hat{C}_3-\hat{T}_1\hat{T}_2-\frac{1}{3!}\hat{T}_1^3 \\
\hat{C}_4 =\hat{T}_4+\hat{T}_1\hat{T}_3+\frac{1}{2}\hat{T}_2^2+\frac{1}{2}\hat{T}_2\hat{T}_1^2+\frac{1}{4!}\hat{T}_1^4 \tab&\tab \hat{T}_4=\hat{C}_4-\hat{T}_1\hat{T}_2-\frac{1}{2}\hat{T}_2^2-\frac{1}{2}\hat{T}_2\hat{T}_1^2-\frac{1}{4!}\hat{T}_1^4
 \end{IEEEeqnarray}

 Hence if we would know the full CI coefficients we can easily extract the corresponding t-amplitudes. This only establishes validity of the parameterization, why is it better? as an approximation? 
 \subsection{Truncated Coupled Cluster}
Let us consider the equations for truncated CC: 
 \begin{IEEEeqnarray}{rLl}
\langle \phi_0|e^T|\phi_0\rangle &= \langle \phi_0|(1+\hat{T}+\frac{1}{2}\hat{T}^2+\cdots )|\phi_0\rangle \notag \\
&= \langle \phi_0| \phi_0\rangle =1 
 \end{IEEEeqnarray}
\tab If exact: 
 \begin{IEEEeqnarray}{rLl}
\langle \phi_0|\hat{H}e^T|\phi_0\rangle =\langle \phi_0|e^T|\phi_0\rangle E=E
 \end{IEEEeqnarray}
\tab The derivation of CC equation: 
 \begin{IEEEeqnarray}{rLl}
\hat{H}e^T|\phi_0\rangle =e^T|\phi_0\rangle E
 \end{IEEEeqnarray}
\tab Multiply by $e^{-\hat{T} }=(e^{\hat{T}})^{-1}$
 \begin{IEEEeqnarray}{rLl}
e^{-\hat{T} }\hat{H}e^T|\phi_0\rangle =e^{-\hat{T} }e^T|\phi_0\rangle E \\
\bar{H} |\phi_0\rangle = |\phi_0\rangle E 
 \end{IEEEeqnarray}
\tab $e^{\hat{T}}$ defines a transformation $u$,$\hat{H}= u^{-1}\hat{H}u$, as $ u^{-1}\hat{H}u$ has same eigenvalues. 
 \begin{IEEEeqnarray}{rLl}
(u^{-1}\hat{H}u)u^{-1}|\phi_ \lambda \rangle =u^{-1}H |\phi_\lambda \rangle = u^{-1}|\phi_\lambda \rangle E_\lambda \\
\bar{H} =e^{-T}\hat{H}e^T =u^{-1}\hat{H}u
 \end{IEEEeqnarray}
\tab The operator $\hat{T}$ is to be defined/ calculated such that $|\phi_0\rangle$, a single determinant, because eigenstate of $\bar{H}$.\\
\tab Note that $\bar{H}$ is non- hermitian as $e^{\hat{T}}$ is not unitary.\\
\tab The equation $e^{-\hat{T}}He^{\hat{T}}|\phi_0\rangle = E|\phi_0\rangle$ can only be solved exactly if $\hat{T}$ contains up to N-fold excitation. To obtain equation we project: 
 \begin{IEEEeqnarray}{rLl}
\langle \phi_0|e^{-\hat{T}} H e^{\hat{T}}|\phi_0\rangle &=\langle \phi_0|\phi_0\rangle =E \\
\langle \phi_a^r|e^{-\hat{T}} H e^{\hat{T}}|\phi_0\rangle &=\langle \phi_a^r|\phi_0\rangle = 0 \\
\langle \phi_{aa}^{rr}|e^{-\hat{T}} H e^{\hat{T}}|\phi_0\rangle &=0 
 \end{IEEEeqnarray}
\tab  project precisely as far as you have operator: $\hat{T}=\hat{T}_1+\hat{T}_2+\hat{T}_3+\cdots$ \\
\tab An intriguing structure emerge: 
 \begin{IEEEeqnarray}{rLl}
 \langle \phi_\lambda |e^{-\hat{T}} H e^{\hat{T}}|\phi_0\rangle =0 \tab \forall \lambda
 \end{IEEEeqnarray}
\tab$\Rightarrow$ Solve for $\hat{T}$ amplitude from these equations: \# equations = \# of amplitudes \\
\tab Once $\hat{T}$ is known: calculate $E= \langle \phi_0|e^{-\hat{T}}He^{\hat{T}}|\phi_0\rangle$ \\
\tab The equation that determine the amplitude do not contain the energy, $E$. This is the main reason CC in size-extensive. 


\subsection{Further analysis CC Equations}
 \begin{IEEEeqnarray}{rLl}
\bar{H} &= e^{-\hat{T}} H e^{\hat{T}} \notag \\
&= H+ [H,T ]+\frac{1}{2} \big[ [H,T],T \big]+\frac{1}{3!} \Big[\big[ [H,T],T \big] ,T\Big] +\frac{1}{n!} \bigg[ \Big[\big[ [H,T],T \big] ,T\Big],T \bigg]
 \end{IEEEeqnarray}
\tab ``A nested commutator expansion'' It terminates exactly at order 4. \\
\tab You can verifying by expanding.
 \begin{IEEEeqnarray}{rLl}
e^{-T}He^T &= (1-T+\frac{1}{2}T^2-\frac{1}{3!}T^3 \cdots )H( 1+T+\frac{1}{2}T^+-\frac{1}{3!}T^3 \cdots) \notag \\
&= H \notag \\
&+ HT-TH \notag \\
&+\frac{1}{2}T^2H +\frac{1}{2}H T^2-THT \notag \\
&+ \cdots \notag \\
&+\frac{1}{2} [(HT-TH)T-T(HT-TH)] \notag \\
&= \frac{1}{2} (HT^2-THT-THT+T^2H )
 \end{IEEEeqnarray}
\tab This nested commutator would go on forever. Why does it terminate after $\frac{1}{4!}$ term?\\
\tab Explanation: \\
\tab $\hat{T}$ contains operator $\hat{r}^\dagger, \hat{s}^\dagger,a,b$ that  excite out of $|\phi_0\rangle$. These operators anticommute: 
\tab Now consider $[H,T] =HT-TH$. Let us use the anticommutation relations to bring the operator in HF to the left.\\
\tab $\langle ij||kl\rangle i^\dagger j^\dagger lk \ t^{rs}_{ab}r^\dagger s^\dagger ba$\\
\tab We would get `contractions' between components of $\hat{H}$ and T.
  \begin{IEEEeqnarray}{rLl}
HT = \wick{\c1 H \ \c1T} +TH
 \end{IEEEeqnarray}
\tab TH: Move all operators without contraction. This yields MO net sign change. Hence, 
  \begin{IEEEeqnarray}{rLl}
HT-TH = \wick{\c1 H \ \c1T} \\ 
\frac{ 1}{2} [\wick{\c1 H \ \c1T},T ] = \frac{1}{2}\big[ [H,T],T \big]
 \end{IEEEeqnarray}
\tab Only contributions survive in which each $T$ is contracted at least once with an operator in $H$. \\
\tab Operators in $T$ $\hat{r}^\dagger\hat{s}^\dagger $, $\hat{a}\hat{b}$ all anticommute, since $r,s \rightarrow$ virtual, $a,b \rightarrow$ occupied. \\
\tab Since $\hat{H}=\hat{h}+\hat{V}$, and $\hat{V}$ contains at most 4 de-excitation operators, i.e. $\langle ba||rs\rangle b^\dagger a^\dagger sr$. Each of these can be calculated against one T-operator, but then it stops.\\
\tab The equations are finite and truncate with 4 operators. \\

There is an other consequence: \\
\tab Every $\hat{T}$ in $(e^{-t}He^T)=\bar{H}=(He^T)_c$ has at least one index in common  with $\hat{H}$. This is indicated by $(He^T)_c$ \\
\tab This nomenclature arises from a diagrammatic representation of the contraction process, so-called  `Feymnman diagrams' (more quantum fioeld ingredients) \\

Let us now return to the extensive question. \\
\tab Two subsystems A, B, non-interacting with each other: \\
  \begin{IEEEeqnarray}{rLl}
HT-TH = \wick{\c1 H \ \c1T} \\ 
\hat{H} = \hat{H}_A +\hat{H}_B 
 \end{IEEEeqnarray}
\tab \tab \tab \tab\tab \tab $\hat{H}_A:$ only integrals with all labels on A.\\
\tab \tab \tab \tab \tab \tab$\hat{H}_B:$ only integrals with all labels on B. \\
\tab All mixed integrals is zero. \\
\tab Then the corresponding $T=T_A+T_B$ and $[ \bar{T}_A,\hat{T}_B ]=0$
  \begin{IEEEeqnarray}{rLl}
e^{-T}He^T & = e^{-(T_A+T_B)}(H_A+H_B)e^{T_A+T_B } \notag \\
&= e^{-T_B}(e^{-T_A}H_Ae^{-T_B})e^{T_B}+ e^{-T_A}(e^{-T_B}H_Be^{-T_B})e^{T_A} \notag \\
&= \bar{H}_A+\bar{H}_B 
 \end{IEEEeqnarray}
\tab A and B have no common indices , also $e^{T_A+T_B }=e^{T_A}e^{T_B}=e^{T_B}e^{T_A}$.\\
\tab It is then easy to demonstrate that $\hat{T}+= \hat{T}_A+\hat{T}_B$ is the solution to the combined set of equations. 
  \begin{IEEEeqnarray}{rLl}
\langle \phi_{rs}^{ab}|\bar{H}_A+\bar{H}_B |\phi_0\rangle =0 
 \end{IEEEeqnarray}
\tab  If \ $\begin{pmatrix}
ab\\rs 
\end{pmatrix} \in A $ : $T_A$ solution \\
\tab  If \ $\begin{pmatrix}
ab\\rs 
\end{pmatrix} \in B $ : $T_B$ solution \\
\tab  If \ $\begin{pmatrix}
ab\\rs 
\end{pmatrix} \in AB $(mixed) : equation vanished automatically. $\bar{H}_A+\bar{H}_B$ does not contain mixed excitations. \\
\tab In CC Theory, the combined AB equation reduces to the isolated subsystem equations. Thus is the basis for locality in physics; This is what we need to solve problems. \\
$\Rightarrow$  Deep physical implication to this expoential structure.

\tab Summary:
\begin{center}
$e^T|\phi_o\rangle = e^{T_A+T_B}|\phi_o\rangle= e^{T_A} +e^{T_B}|\phi_0\rangle \approx \psi_A+\psi_B  $	
\end{center} 
The wavefunction takes the product form:
\begin{center}
$E=\langle \phi_0|\bar{H}_A+\bar{H}_B|\phi_0\rangle =E_A+E_B $	
\end{center} 
$\Rightarrow$ Size-consistency is build into the theory.\\

\subsection{Apply in Practice}
How does CC theory work in practice? 
\begin{itemize}
	\item CCSD, \ $n_O^2 n_v^4$ scaling is fairly efficient, but is not quantitively accurate.
	\item CCSDT, $n_o^3n_v^5$ prohibitively expensive only used for benchmarking (or very high accuracy)
	\item CCSD(T), $n_o^2n_v^4$ iteratively (like CCSD), $n_0^3n_v^4$ non-iterative.
\end{itemize}
\tab Add a perturbative correlation to CCSD. The (T) step is most expensive part typically, but still practice.

CCSD(T) in combination with large basis sets yields accurate results (for single reference situations)
\begin{itemize}
	\item geometries
	\item vibrational frequencies
	\item thermochemistry
	\item reaction barriers (usually ok)
\end{itemize}
\tab Still, very expensive because of basis set requirements

CCSD(T) is often referred to as good standard in quantum chemistry. However, this method cannot deal with general multireference situations:
\begin{itemize}
	\item bond breaking
	\item biradical
	\item transition metal sysrtems
\end{itemize}

CC methods for excited states \\
\tab A straightforward extension allows the calculation of excited states using coupled cluster theory.\\
\tab Procedures:\\
\tab Solve CCSD equations for the ground state\\
\tab Obtain transformed Hamiltonian $\bar{H}=e^{-T}+e^T$. This $\bar{H}$ is diagonalized single and double excited configuration. \\
\begin{center}
\begin{tabular}{|c|c|c|c|c|}
\hline 
$\hat{H}$ & O & S & D & T \\
\hline  
0 & $E_0$ & X & X &   \\
\hline 
S & O & X & X &  \\
\hline 
D & O & X & X & \\
\hline 
T & X & $\sim$ & X &\\
\hline
\end{tabular} \\	
\end{center}
\tab X: sizeable matrix-elements \\
\tab O: rigorously O \\
\tab $\sim$: small remainder

  \begin{IEEEeqnarray}{rLl}
\begin{pmatrix}
A \ B\\ O \ D
\end{pmatrix}
\begin{pmatrix}
1 \\ 0
\end{pmatrix} &= 
\begin{pmatrix}
A \\O
\end{pmatrix} \\
\begin{pmatrix}
A \ B\\ O \ D
\end{pmatrix}
\begin{pmatrix}
x \\ y
\end{pmatrix} &= 
\begin{pmatrix}
Ax+ By \\ Dy
\end{pmatrix} = E_\lambda 
\begin{pmatrix}
x \\ y
\end{pmatrix} 
 \end{IEEEeqnarray}

$Dy = E_\lambda y$: diagonalize  $S+D$ block. Does not couple to groud state.\\
\tab This method is called Equation-of-motion coupled cluster or Coupled Cluster linear response theory or (Closely related) SAC-CI. \\
\tab SAC-CI was first. Now in gaussian.\\
\tab Accuracy EOM-CCSD\\
\tab $\sim$ 0.1 eV for Rydberg states \\
\tab $\sim$ 0.3 eV for Valeru states \\
Only works for singly excited states, because ``$\sim$'' in $\bar{H}$ matrix.  small $S\rightarrow T$ block.\\
\tab Calculation of IP's  and EA's: ($E_N-E_{N-1}$ and $E_{N+!}-E_N$\\
\tab Essentially same procedure! \\
\tab Construct $\bar{H}$ elements over ionized or electron-attached states 
\begin{table}[!htbp]
\centering
\begin{tabular}{|c|c|c|}
\hline 
$\bar{H}$ & $1h$ & $2h1p$ \\
\hline  
$1h$ & X & X  \\
\hline 
$2h1p$ & X & X \\
\hline 
\end{tabular}
\end{table}\\
\tab $\sim$ 0.1-0.2 eV accuracy for Koopman's states \\
\tab $1h$: $\hat{a}|HF\rangle$\\
\tab $2h1p:$ $\hat{a}\hat{r}^\dagger\hat{b}|HF\rangle$\\
\tab Minimal set to get good accuracy.











\end{document}