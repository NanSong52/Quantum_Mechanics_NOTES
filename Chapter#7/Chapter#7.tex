\documentclass[a4paper, 12pt]{article}
\usepackage{matnoble-doc-en}
\usepackage{amsmath}
\usepackage{simpler-wick}
\newcommand\Item[1][]{%
  \ifx\relax#1\relax  \item \else \item[#1] \fi
  \abovedisplayskip=0pt\abovedisplayshortskip=0pt~\vspace*{-\baselineskip}}


\begin{document}

\title{\bf {CHEM400/740: Quantum Mechanics in Chemistry\\ Chapter\#07: Second Quantization}} \author{\bf
  \href{http://scienide2.uwaterloo.ca/~nooijen/website_new_20_10_2011/About.html}{Marcel Nooijen}} \date{}
  
\pagestyle{fancy} \fancyhead[L]{\textcolor{PrimaryColor}{CHEM400/740: Quantum Mechanics in Chemistry}} \fancyhead[R]{\textcolor{PrimaryColor}{2021 Winter}}


\maketitle
\tableofcontents

\clearpage


\section{Review Session}
\subsection{Operator Approach to Vibronic Problems}
Operators: $\hat{b}_i$, $\hat{b}_j^\dagger$
\begin{IEEEeqnarray}{rLl}
(\hat{b}_i)^\dagger = \hat{b}_i^\dagger
\end{IEEEeqnarray}

\begin{itemize}
	\item Satisfy commutation relations:
\begin{IEEEeqnarray}{rLl}
\left[ \hat{b}_i,\hat{b}_j \right] &= \left[ \hat{b}_i^\dagger, \hat{b}_j^\dagger \right] =0 \\
\left[ \hat{b}_i,\hat{b}_j^\dagger \right] &=\delta_{ij}
\end{IEEEeqnarray}
	\item States were expressed as:
\begin{IEEEeqnarray}{rLl}
|n_i,n_j \cdots \rangle = \frac{(\hat{b}_i^\dagger )^{n_i}}{\sqrt{n_i!}}\frac{(\hat{b}_j^\dagger )^{n_j}}{\sqrt{n_j!}} \cdots| 0\rangle
\end{IEEEeqnarray}
NOTE:  only are state $|0\rangle$ rest is expressed in terms of operators. $\langle 0|(\hat{b}_i^\dagger )^k (\hat{b}_j)^l |0\rangle$
	\item Operators we expressed using powers of $\hat{b}_i^\dagger, \hat{b}_j$.
	\item The vacuum state has the property that: 
 \begin{IEEEeqnarray}{rLl}
\hat{b}_i|0\rangle = \langle 0|\hat{b}_i^\dagger =0
\end{IEEEeqnarray}

	\item Everything reduces to algebra, manipulating commutation relations.
	\item Wavefunction operators were expressed in terms of $\hat{b}_i^\dagger, \hat{b}_i $
\end{itemize}


Question 1: Can we do something similar for the electronic structure problem?

Goal: Reduce the problem to algebra. 
\begin{itemize}
	\item[-]  Suitable for computers.
	\item[-]  We can design creative methods, not easily formulated using wave functions.
\end{itemize}


Ingredients: 
\begin{itemize}
\item[1)] Take ordered set of orbitals $\{ \psi_a,\psi_b,\psi_c\cdots \psi_M \}$.
\item[2)] Specify Slater determinant by occupation number 0 or 1 for each orbital.\\
$|n_a,n_b,n_c,\cdots,n_m \rangle$ for $n_i=0$ or $n_i=1$ \\
\fbox{Analogy: normal mode $i \leftrightarrow$ orbital $a$ }\\
Example: $|1_a 0_b 1_c \cdots 1_z\rangle \equiv \frac{1}{\sqrt{N!}}|\psi_a(1)\psi_c(2)\cdots \psi_z(N) |$\\
\tab  The sign is uniquely determined because the ordering of orbitals in the determinant is fixed.
\item[3)] The total number of determinants, $M$, can make  $2^M$ states:\\
Every orbital can either be occupied or not $\rightarrow \overbrace{ (2 \times 2\times \cdots \times 2 )}^{M \text{ times}}$ 
\item[4)] This structure includes states containing from 0 to $M$ electrons. This space of basis -vector is called '' Fock-space''.
\item[5)] In order to define a Slater determinant in the formalism of second quantization, we introduce a vacuum state with 0 electrons.
\begin{IEEEeqnarray}{rLl}
|vac \rangle = |0,0,\cdots,0 \rangle
\end{IEEEeqnarray}
Moreover, this state is normalized, as are all basis states.
\begin{IEEEeqnarray}{rLl}
\langle vac|vac \rangle = \langle 0|0\rangle =1
\end{IEEEeqnarray}
This indicates the structure is richer than simple `` Slater determinants'' The vacuum plays a crucial role, and is not an insignificant detail. 
\item[6)] Basis states can be written as occupation number vector $|n\rangle$, or $|\bar{n}\rangle$. 
\begin{IEEEeqnarray}{rLl}
| \bar{n} \rangle = | n_a, n_b,\cdots n_M \rangle
\end{IEEEeqnarray}
\item [7)] Similarly as for vibrations we introduce annihilation and creation operators here. Summarize their properties they have, and justify/derive later.
\begin{IEEEeqnarray}{rLl}
\hat{a}_p^\dagger |\bar{n}_p \rangle &\rightarrow |\bar{n}+1_p\rangle \cdot \text{sign} \\
\hat{a}_p|\bar{n}_p\rangle &\rightarrow |\bar{n}-1_p\rangle \cdot \text{sign}
\end{IEEEeqnarray}
NOTE: 
\begin{itemize}
\item[a.] $\bar{n}+1_p$ : raise $n_p$ by 1.
\item[b.]  $\bar{n}-1_p$: lower $n_p$ by 1.
\item[c.] All other $n_i$ stay the same.
\end{itemize}
\begin{IEEEeqnarray}{rLl}
\hat{a}_p|vac\rangle &= \langle vac| \hat{a}_p^\dagger =0 \\
(\hat{a}_p)^\dagger &= \hat{a}_p^\dagger
\end{IEEEeqnarray}
The operators satisfy anticommutation relations: 
\begin{IEEEeqnarray}{rLl}
\hat{a}_i \hat{a}_j + \hat{a}_j\hat{a}_i &=0\\
\hat{a}_i^\dagger \hat{a}_j^\dagger + \hat{a}_j^\dagger \hat{a}_i^\dagger &=0 \\
\hat{a}_i^\dagger \hat{a}_j + \hat{a}_j \hat{a}_i^\dagger &= \delta_{ij}
\end{IEEEeqnarray} 
For $i=j$: 
\begin{IEEEeqnarray}{rLl}
\hat{a}_i^\dagger \hat{a}_i^\dagger + \hat{a}_i^\dagger \hat{a}_i^\dagger &=0 \\
\Rightarrow \hat{a}_i^\dagger \hat{a}_i^\dagger &=0
\end{IEEEeqnarray} 
This solution is to build a formal structure. I will not proceed formally but try to show what it means.
\end{itemize}






Question 2:  Role of vacuum state, $|0 \rangle$, in vibrational V.S. electronic problem. 
\begin{itemize}
	\item [1)] Vibrational problem: \\
Strict identity of wave function/ coordinate space representation, and second quantitation.
\begin{IEEEeqnarray}{rLl}
|0\rangle = Ne^{-\frac{1}{2}q^2}
\end{IEEEeqnarray}
$(\hat{b}^\dagger )^n|0\rangle$ forms a new function. If there are 2 normal modes, $q_1, q_2$, hence:
\begin{IEEEeqnarray}{rLl}
|0\rangle_1 |0\rangle_2 = Ne^{-\frac{1}{2}(q_1^2+q_2^2)}
\end{IEEEeqnarray}
$(b_1^\dagger )^n|0\rangle_1, (b_2^\dagger )^k|0\rangle_2$ produce nothing new, but more convenient.
	\item [2)] Electronic problem:\\
Something rarely new: \\
Consider states with differing number of electron.\\
Full antisymmetric: $|n_1,n_2,n_3 \cdots \rangle$ \tab $n_i =$ 0 or 1 
\begin{IEEEeqnarray}{rLl}
|0,0,0\cdots \rangle &= |0 \rangle \\
\langle 0|0\rangle &= 1
\end{IEEEeqnarray}
Why normalize to 1? For product $|0_A\rangle |0_B\rangle$:
\begin{IEEEeqnarray}{rLl}
\langle 0_A|0_A\rangle \langle 0_B|0_B\rangle = \langle 0|0 \rangle=1
\end{IEEEeqnarray}
\end{itemize}


\section{Second Quantization}
\subsection{Informal Introduce of Annihilation and Creation Operators }
\subsubsection{Creation Operator}
Creation Operator, $\hat{a}_p^\dagger$:
\begin{IEEEeqnarray}{rLl}
\hat{a}_p^\dagger |\psi_a\psi_b\cdots \psi_z | &= |\psi_p\psi_a\cdots \psi_z | = | \bar{n} \rangle \notag \\
&= (-)^{\Gamma_{p,\bar{n}}} |\psi_a\psi_b\cdots \psi_p\cdots \psi_z | 
\end{IEEEeqnarray} 
\begin{itemize}
\item[1)] Add orbital $\psi_p$ to the first position, using the ordering of the orbitals.
\item[2)] This results a sign change. \\
$\Gamma _{p,\bar{n}}$: the number of occupied orbitals before the location of $\psi_p$.
\begin{IEEEeqnarray}{rLl}
\hat{a}_p^\dagger |\bar{n}\rangle &= (-)^{\Gamma_{p,\bar{n}}} | \bar{n}+1_p\rangle \cdot (1-n_p)
\end{IEEEeqnarray} 
NOTE: 
\begin{itemize}
\item In original $|\bar{n}\rangle$, $|\bar{n} = |n_1,n_2,\cdots,n_M\rangle$.\\ M represents the size of basis.
\item Consider the term of $(1-n_p)$.

 If $\bar{n}_p=0$, $p$ orbital is unoccupied in original $|\bar{n}\rangle$.
 \begin{IEEEeqnarray}{rLl}
\hat{a}_p^\dagger |\bar{n}\rangle &= (-)^{\Gamma_{p,\bar{n}}} | \bar{n}+1_p\rangle \end{IEEEeqnarray}  

If $\bar{n}_p=1$, $p$ orbital is occupied.
 \begin{IEEEeqnarray}{rLl}
 \hat{a}_p^\dagger |\bar{n}\rangle &= (1-n_p) (-)^{\Gamma_{p,\bar{n}}} | \bar{n}+1_p\rangle  \notag \\
 &= 0
\end{IEEEeqnarray}  
This sign is indeed to incorporate antisymmetry.
\end{itemize}

\item[3)] $\hat{a}_p^\dagger |\hat{n}\rangle$ contains one more electron than $|\bar{n}\rangle$ (or is zero).
\end{itemize}
\subsubsection{ Annihilation Operator}
Annihilation Operator, $\hat{a}_p$:
\begin{IEEEeqnarray}{rLl}
\hat{a}_p|\bar{n}\rangle &= (-)^{\Gamma_{p,\bar{n}}}|\bar{n}-1_p\rangle \cdot n_p
\end{IEEEeqnarray}
\tab Consider the term of $n_p$:
\begin{itemize}
	\item[1)] If $n_p=0$, $p $ orbital is unoccupied.
\begin{IEEEeqnarray}{rLl}
\hat{a}_p|\bar{n}\rangle &= (-)^{\Gamma_{p,\bar{n}}}|\bar{n}-1_p\rangle \cdot n_p \notag \\
&= 0
\end{IEEEeqnarray}
	\item[2)] If $n_p=1$, $p$ orbital is occupied in $|\bar{n}\rangle$. 
\begin{IEEEeqnarray}{rLl}
\hat{a}_p|\bar{n}\rangle &= (-)^{\Gamma_{p,\bar{n}}}|\bar{n}-1_p\rangle \\
\hat{a}_p|\psi_a\cdots \psi_p\cdots \psi_z | &= \hat{a}_p(-)^{\Gamma_{p,\bar{n}}} |\psi_p\psi_a\cdots \psi_z | \\
 &\equiv (-)^{\Gamma_{p,\bar{n}}} |\psi_a\psi_b\cdots \psi_z | 
\end{IEEEeqnarray}
Action of $\hat{a}_p$:\\
 Bring orbital to first position (sign change) then remove it from determinant. Contains one less electron than original $|\bar{n}\rangle$.
\end{itemize}

\subsubsection{Anticommutation Relations}
Commutation:
\begin{IEEEeqnarray}{rLl}
\left[ \hat{A},\hat{B} \right]=\hat{A}\hat{B}-\hat{B}\hat{A}
\end{IEEEeqnarray}
\tab Anticommutation: 
\begin{IEEEeqnarray}{rLl}
\left[ \hat{A},\hat{B} \right]_+ \equiv \{\hat{A},\hat{B}\} \equiv \hat{A}\hat{B}+\hat{B}\hat{A}
\end{IEEEeqnarray}
\tab From these definition we can deduce the anticommutation relations: 
\begin{IEEEeqnarray}{rLl}
\hat{a}_p^\dagger \hat{a}_q^\dagger + \hat{a}_q^\dagger \hat{a}_p^\dagger &=0 \\
\hat{a}_p \hat{a}_q+ \hat{a}_q\hat{a}_p &=0 \\
\hat{a}_p^\dagger \hat{a}_q+ \hat{a}_q \hat{a}_p^\dagger &= \delta_{pq} = \left\{ \begin{array}{rcl}
        1 & p=q \\
        0 & p\neq q 
           \end{array}\right.
\end{IEEEeqnarray}

Demonstration: 
\begin{itemize}
	\item $(\hat{a}_p^\dagger \hat{a}_q^\dagger + \hat{a}_q^\dagger \hat{a}_p^\dagger) |\psi_a\psi_b\cdots\psi_z| \\
	= \hat{a}_p^\dagger|\psi_q\psi_a\cdots\psi_z|+\hat{a}_q^\dagger |\psi_p\psi_a\cdots\psi_z| \\
	= |\psi_p \psi_q\psi_a\cdots\psi_z|+|\psi_q \psi_p\psi_a\cdots\psi_z| \\
	= |\psi_p \psi_q\psi_a\cdots\psi_z|-|\psi_p \psi_q\psi_a\cdots\psi_z| \\
	=0 $
	\item $ (\hat{a}_p \hat{a}_q+ \hat{a}_q\hat{a}_p ) |\psi_a\cdots \psi_p\cdots\psi_q \cdots \psi_z| \\
	= (\hat{a}_p \hat{a}_q+ \hat{a}_q\hat{a}_p) (sign)|\psi_p \psi_q\psi_a\cdots\psi_z|\\
	=(sign) (\hat{a}_p \hat{a}_q (-) |\psi_q \psi_p\psi_a\cdots\psi_z|+\hat{a}_q\hat{a}_p |\psi_p \psi_q\psi_a\cdots\psi_z| )\\
	=(sign)(-|\psi_a\psi_b\cdots\psi_z|+|\psi_a\psi_b\cdots\psi_z|)\\
	=0  $\\
	If determinant dose not contain $\psi_p$ or $\psi_q$, every contribution is 0.  Hence,\\
	$(\hat{a}_p \hat{a}_q+ \hat{a}_q\hat{a}_p)|\bar{n}\rangle = 0 \tab \forall \bar{n}\\
	(\hat{a}_p \hat{a}_q+ \hat{a}_q\hat{a}_p)=0$\\
	I use here also that $\hat{a}_p$ is a linear operator. More discussion later.
	\item For $p\neq q$:\\
	$ (\hat{a}_p^\dagger \hat{a}_q+ \hat{a}_q \hat{a}_p^\dagger )|\psi_a\cdots \psi_q\cdots \psi_z | \left\{ \begin{array}{rcl}
        n_q=1 \\
        n_p=0 
           \end{array}\right. $\\
    $ = (sign)(\hat{a}_p^\dagger \hat{a}_q+ \hat{a}_q \hat{a}_p^\dagger)|\psi_q\psi_a\cdots  \psi_z | \\
    = (sign)(|\psi_p \psi_a\cdots\psi_z |+\hat{a}_q|\psi_p \psi_q\psi_a\cdots\psi_z| )\\
    = (sign)( |\psi_p \psi_a\cdots\psi_z |- |\psi_p \psi_a\cdots\psi_z|)\\
    =0 $\\
    If not $n_p=1, n_q=0 $, every term is 0. Hence,\\
    $\hat{a}_p^\dagger \hat{a}_q+ \hat{a}_q \hat{a}_p^\dagger  =0 \tab p\neq q  $\\
	\\
	For $p=q$:\\
	$(\hat{a}_p^\dagger \hat{a}_p+ \hat{a}_p \hat{a}_p^\dagger)|\psi_a\cdots\psi_p\cdots\psi_z | \tab n_p=1 \\
	=(\hat{a}_p^\dagger \hat{a}_p+ \hat{a}_p \hat{a}_p^\dagger )(sign)|\psi_p\psi_a\cdots\psi_z |\\
	=\hat{a}_p^\dagger |\psi_a\cdots \psi_z|(sign)+0 \\
	=| \psi_p\psi_a\cdots\psi_z |(sign)\\
	=|\psi_a\cdots\psi_p\cdots\psi_z |   $
	\\
	\\
	$(\hat{a}_p^\dagger \hat{a}_p+ \hat{a}_p \hat{a}_p^\dagger))|\psi_a\cdots\psi_z | \tab n_p=0\\
	=0+\hat{a}_p |\psi_p\psi_a\cdots\psi_z | \\
	= |\psi_a\cdots\psi_z |  $\\
	\\
	$\Rightarrow (\hat{a}_p^\dagger \hat{a}_p+ \hat{a}_p \hat{a}_p^\dagger)|\bar{n}\rangle =|\bar{n}\rangle \tab \forall |\bar{n}\rangle \\
	\Rightarrow \hat{a}_p^\dagger \hat{a}_p+ \hat{a}_p \hat{a}_p^\dagger = 1  $
	
\end{itemize}
We further stipulate that the operator $\hat{a}_p$ and $\hat{a}_p^\dagger$ are linear.
\begin{IEEEeqnarray}{rLl}
\hat{a}_p (\sum_{\bar{n}}|\bar{n}\rangle c_n ) &= \sum_{\bar{n}} (\hat{a}_p|\bar{n}\rangle )c_n \\
\hat{a}_p^\dagger (\sum_{\bar{n}}|\bar{n}\rangle c_n ) &= \sum_{\bar{n}} (\hat{a}_p^\dagger |\bar{n}\rangle )c_n
\end{IEEEeqnarray}
Then also products, and linear combination are linear operators.
\begin{IEEEeqnarray}{rLl}
\sum_{p,q,r} \chi_{pqr} \hat{a}_p^\dagger \hat{a}_q \hat{a}_r (\sum_{\bar{n}}|\bar{n}\rangle c_{\bar{n}} ) = \sum_{p,q,r} \chi_{pqr} \sum_{\bar{n}} (\hat{a}_p^\dagger\hat{a}_q\hat{a}_r|\bar{n}\rangle ) c_n
\end{IEEEeqnarray}
NOTE:
\begin{itemize}
	\item [1)] $ \hat{a}_p^\dagger\hat{a}_q\hat{a}_r|\bar{n}\rangle$ is well-defined using definitions.
	\item [2)] Just evaluate one piece at a time, in whatever order, and sum it.
	\item [3)] This is in analogy with the boson operators $b_i^\dagger, b_j, \cdots$\\
	$b_i^\dagger b_j^\dagger -b_j^\dagger b_i^\dagger =0$ Bosons: integer spin\\
	$ \hat{a}_p^\dagger \hat{a}_q^\dagger + \hat{a}_q^\dagger \hat{a}_p^\dagger = 0$ Fermions: half-integer spin 
\end{itemize}
Every Slater determinant can be written as:
\begin{IEEEeqnarray}{rLl}
|\bar{n}\rangle &= |\psi_a\psi_b\cdots \psi_z | \notag \\
&= (\hat{a}_a^\dagger \hat{a}_b^\dagger \cdots \hat{a}_z^\dagger )| vac\rangle \notag \\
&=  (\hat{a}_a^\dagger \hat{a}_b^\dagger \cdots \hat{a}_z^\dagger ) |\psi_z|  \notag \\
&= \hat{a}_a^\dagger |\psi_b\cdots \psi_z | \\
\langle \bar{n} | &= ((\hat{a}_a^\dagger \hat{a}_b^\dagger \cdots \hat{a}_z^\dagger )|vac\rangle )^\dagger \notag \\
&= \langle vac|\hat{a}_z\cdots\hat{a}_b\hat{a}_a
\end{IEEEeqnarray}
Note that $(\hat{a}_a^\dagger )^\dagger=\hat{a}_a $.\\
Take Hermitian conjugate and reverse order of operators. 
\begin{IEEEeqnarray}{rLl}
\langle \bar{m} |\hat{a}_p^\dagger |\bar{n} \rangle &= \langle \bar{n} |\hat{a}_p |\bar{m} \rangle ^* \\
\langle n_a\cdots 1_p\cdots n_M|\hat{a}_p^\dagger|n_a\cdots0_p \cdots n_M\rangle &= \langle n_a\cdots 0_p\cdots n_M|\hat{a}_p|n_a\cdots 1_p \cdots n_M\rangle \notag \\
&= 1
\end{IEEEeqnarray}
Other combinations of $n_p,m_p \rightarrow 0$.\\
Operators are Hermitian conjugate: $(\hat{a}_p)^\dagger = \hat{a}_p^\dagger$. \\
\tab The analogy with bosons operators is complete if we can express any operator in terms of $\hat{a}_p^\dagger, \hat{a}_p$. (i.e. linear combinations and powers) This is indeed the case and moreover these expressions can be very compact. 

\subsection{Second-Quantized Operators and Their Matrix Elements}
\subsubsection{The Hamiltonian in Second Quantization}
How to express operators in second quantization language?\\
\tab The Hamiltonian can be expressed as:
\begin{IEEEeqnarray}{rLl}
\hat{H} &= \sum_{p,q}\langle p|\hat{h}|q \rangle  \hat{a}_p^\dagger \hat{a}_q + \frac{1}{4} \sum_{p,q,r,s} \langle pq||rs\rangle \hat{a}^\dagger_p\hat{a}^\dagger_q\hat{a}_s \hat{a}_r \notag \\
&= \sum_{p,q} \hat{h}_{pq} \hat{a}_p^\dagger \hat{a}_q + \frac{1}{4} \sum_{p,q,r,s} \langle pq||rs\rangle \hat{a}^\dagger_p\hat{a}^\dagger_q\hat{a}_s \hat{a}_r
\end{IEEEeqnarray}
\begin{itemize}
	\item[1)] This Hamiltonian is expresses directly in terms of the one- and two- electron integrals.\\
	 One-electron integral: $  \hat{h}_{pq} = \langle p|\hat{h}|q \rangle = \hat{T}+\hat{V}^{Ne}$.\\
	 Two-electron integral: $ \langle pq||rs\rangle= \langle pq|rs\rangle -\langle pq|sr\rangle$. (spin-orbitals)
	 \item[2)] This Hamiltonian is valid for any number of electrons.This is to be compared to: \\
	$\hat{H}= \sum_{\lambda,\mu} |\phi_\lambda \rangle\langle \phi_\lambda|\hat{H}|\phi_\mu\rangle \langle \phi_\mu| $ \\
	Here, sum over all determinants with $N$ electrons. \\
	Alternative form:\\
	$\frac{1}{4} \sum_{p,q,r,s} \hat{p}^\dagger\hat{q}^\dagger\hat{s}\hat{r} \langle pq||rs\rangle= \sum_{p<q,r<s} \langle pq||rs\rangle \hat{p}^\dagger\hat{q}^\dagger\hat{s}\hat{r}$\\
	And/ or:\\
	 $\frac{1}{2} \sum_{p,q,r,s}\langle pq|rs\rangle  \hat{p}^\dagger\hat{q}^\dagger\hat{s}\hat{r} $
\end{itemize}

Note that I am changing notation. Once it is understood $\hat{a}_p,\hat{a}_p^\dagger$ refers to fermion operators. It is much more convenient to simplify $\hat{p} = \hat{a}_p$, $\hat{p}^\dagger=\hat{a}_p^\dagger $. We can even abbreviate further and simplify write, $p=\hat{a}, q^\dagger =\hat{q}^\dagger $.

Let us consider alternatives: 
\begin{IEEEeqnarray}{rLl}
& \frac{1}{4} \sum_{p,q,r,s} \langle pq||rs\rangle p^\dagger q^\dagger s r \notag \\
=&  \frac{1}{4} \sum_{p,q,r,s} (\langle pq|rs\rangle-\langle pq|sr\rangle) p^\dagger q^\dagger s r \notag \\
=&  \frac{1}{4} \sum_{p,q,r,s} \langle pq|rs\rangle p^\dagger q^\dagger s r  - \frac{1}{4} \sum_{p,q,r,s}   \langle pq|sr\rangle p^\dagger q^\dagger s r \tab  \text{Reverse order} \rightarrow -(sign) \notag \\
=&   \frac{1}{4} \sum_{p,q,r,s} \langle pq|rs\rangle p^\dagger q^\dagger s r  + \frac{1}{4} \sum_{p,q,r,s}   \langle pq|sr\rangle p^\dagger q^\dagger rs \tab \text{Dummy Summation, rename }r\leftrightarrow s, s\leftrightarrow r  \notag \\
=&  \frac{1}{2} \sum_{p,q,r,s} \langle pq|rs\rangle p^\dagger q^\dagger sr
\end{IEEEeqnarray}


Let us demonstrate (argue) that hamiltonian in second quantization is indeed correct. \\
\tab $\Rightarrow$ yields same matrix-elements as original hamiltonian.(Slater Rules)
\begin{IEEEeqnarray}{rLl}
\left[ \sum_{p,q}h_{pq}p^\dagger q ,\hat{r}^\dagger \right] &= \left[ \hat{h},\hat{r}^\dagger \right] = \hat{h}r^\dagger -r^\dagger\hat{h} \\
\hat{h}r^\dagger &=   \sum_{p,q}h_{pq}p^\dagger q r^\dagger  \notag \\
&=   \sum_{p,q}h_{pq}p^\dagger \{q, r^\dagger \} -  \sum_{p,q}h_{pq}p^\dagger r^\dagger q \notag \\
&=  \sum_{p,q}h_{pq}( p^\dagger \delta_{qr} +  r^\dagger p^\dagger q )\notag \\
&= \sum_p h_{pr}p^\dagger +r^\dagger\hat{h}
\end{IEEEeqnarray}
\tab Or:
\begin{IEEEeqnarray}{rLl}
\left[ \hat{h},\hat{r}^\dagger \right] &= \sum_p h_{pr}p^\dagger +r^\dagger\hat{h} - r^\dagger\hat{h} \notag \\
&=\sum_p h_{pr}p^\dagger
\end{IEEEeqnarray}

\subsubsection{The Matrix Elements in Second Quantization}
Now evaluate matrix elements of one-electron integral: 
\begin{IEEEeqnarray}{rLl}
\langle K|\hat{h}|K\rangle &= \langle vac|(\hat{z}\cdots \hat{b}\hat{a} ) \ \hat{h} \ \hat{a}^\dagger\hat{b}^\dagger \cdots \hat{z}^\dagger |vac\rangle \notag \\
&= \langle vac|(\hat{z}\cdots \hat{b}\hat{a} )\left[ \hat{h},a^\dagger \right] \hat{b}^\dagger \cdots \hat{z}^\dagger |vac\rangle \notag \\
&+\langle vac|(\hat{z}\cdots \hat{b}\hat{a} )\hat{a}^\dagger \left[ \hat{h},\hat{b}^\dagger \right]  \cdots \hat{z}^\dagger |vac\rangle \notag \\
&+ \cdots \notag \\
&+\langle vac|(\hat{z}\cdots \hat{b}\hat{a} )\hat{a}^\dagger \hat{b}^\dagger \cdots \left[ \hat{h},\hat{z}^\dagger \right]  |vac\rangle \notag \\
&+ \langle vac|(\hat{z}\cdots \hat{b}\hat{a} )(\hat{a}^\dagger \hat{b}^\dagger \cdots \hat{z}^\dagger )\hat{h}  |vac\rangle \notag \\
&= h_{aa}+h_{bb} +\cdots + h_{zz} 
\end{IEEEeqnarray}
\tab In order to give non-zero matrix-elements $\left[ h, a^\dagger \right] b^\dagger \cdots z^\dagger|vac\rangle \rightarrow \langle ab\cdots z|$: 
\begin{center}
	$\langle vac|\hat{a} \left[ \hat{h},a^\dagger \right] | vac\rangle = \langle vac|\hat{a} \sum_p h_{pa}p^\dagger | vac\rangle $
\end{center}
\tab  Hence, only surviving contribution, $\left[ h, a^\dagger \right] =\sum_p h_{pa}p^\dagger$, when $p=a$ only contribution. \\
\tab Same argument for $\left[ \hat{h},b^\dagger \right],...\left[ \hat{h},z^\dagger \right]$.\\
\tab $\Rightarrow$ Contribution: $h_{aa}+h_{bb}+\cdots+h_{zz}$.

Final contribution: 
$a^\dagger b^\dagger\cdots z^\dagger h_{pq} \hat{p}^\dagger \hat{q}|vac \rangle = 0 $, because $\hat{q}|vac\rangle =0 $. \\
\tab Hence, $ \langle K|h|K\rangle= \sum_{a \text{ in } K} h_{aa} $ is correct result.

I used a useful tool:
\begin{IEEEeqnarray}{rLl}
\hat{h}\hat{O}_1\hat{O}_2\hat{O}_3 &= \left[ \hat{h},\hat{O}_1 \right]\hat{O}_2\hat{O}_3+\hat{O}_1 \hat{h} \hat{O}_2\hat{O}_3 \notag \\
&= \left[ \hat{h},\hat{O}_1 \right]\hat{O}_2\hat{O}_3+ \hat{O}_1\left[ \hat{h},\hat{O}_2 \right]\hat{O}_3+\hat{O}_1\hat{O}_2\left[ \hat{h},\hat{O}_3 \right] + \hat{O}_1\hat{O}_2\hat{O}_3 \hat{h}
\end{IEEEeqnarray}
\tab This is true for any string of operators.
\begin{IEEEeqnarray}{rLl}
\left[ \hat{O},\hat{A}\hat{B}\hat{C}\hat{D}\right] &= \hat{O}\hat{A}\hat{B}\hat{C}\hat{D}-\hat{A}\hat{B}\hat{C}\hat{D}\hat{O}\\
\left[ \hat{O},\hat{A}\hat{B}\hat{C}\hat{D}\right] &= \left[ \hat{O},\hat{A}\right] \hat{B}\hat{C}\hat{D}+\hat{A}\left[ \hat{O},\hat{B}\right] \hat{C}\hat{D} +\hat{A}\hat{B}\left[ \hat{O},\hat{C}\right]\hat{D}+\cancel{ \hat{A}\hat{B}\hat{C}\hat{D}\hat{O}}
\end{IEEEeqnarray}
\tab Let us also do off-diagonal one-electron matrix element: ($z,b,\cdots,p\neq a$)
\begin{IEEEeqnarray}{rLl}
\langle K'|\hat{h}|K\rangle &= \langle vac|(z\cdots b p) \ \hat{h} \ (a^\dagger b^\dagger \cdots z^\dagger )|vac\rangle \notag \\
 &= \langle vac|((z\cdots b p) \left[ \hat{h},a^\dagger \right] \hat{b}^\dagger \cdots \hat{z}^\dagger |vac\rangle \notag \\
&+\langle vac|(z\cdots b p) \hat{a}^\dagger \left[ \hat{h},\hat{b}^\dagger \right]  \cdots \hat{z}^\dagger |vac\rangle \notag \\
&+ \cdots \notag \\
&= \langle vac|((z\cdots b p) \left[ \hat{h},a^\dagger \right] \hat{b}^\dagger \cdots \hat{z}^\dagger |vac\rangle \notag \\
&= h_{pa}
\end{IEEEeqnarray}
\tab The last term and all others vanish because orbital $a$ (from $a^\dagger$) is not among $\langle vac|\hat{z}\cdots \hat{b}\hat{p}$.\\
\tab Only the first term contributes: $\left[ \hat{h},a^\dagger \right] =\sum_rh_{ra}r^\dagger \rightarrow r=p $.\\
\tab Hence, $ \langle K'|\hat{h}|K\rangle=  h_{pa} $, as before.

Note: We do not need to worry about permutations or factor of $N!$. Annihilation and creation operators take care of it all.

Let us do some more matrix-element, involving two-electron integral ($\hat{V}= \frac{1}{4} \sum_{p,q,r,s} \langle pq||rs\rangle p^\dagger q^\dagger s r $).\\
\tab Consider $\langle K''|V|K\rangle $:
\begin{IEEEeqnarray}{rLl}
\langle K''|V|K\rangle &= \langle vac|(z\cdots cut )(\sum\frac{1}{4}p^\dagger q^\dagger s r  \langle pq||rs\rangle )a^\dagger b^\dagger c^\dagger \cdots z^\dagger|vac\rangle 
\end{IEEEeqnarray}
\tab $p, q$ orbitals try to match $t,u$ orbitals, $(p,q) \leftrightarrow (t,u) $ \\
\tab $r, s$ orbitals try to match $a,b$ orbitals, $(r,s) \leftrightarrow (a,b) $ \\
\tab Hence, only contributions: 
\begin{itemize}
	\item $s=b, r=a, p=t, q=u \tab +\langle tu||ab\rangle \frac{1}{4}$
	\item $s=a,r=b,p=t,q=u \tab -\langle tu||ba\rangle \frac{1}{4}$
	\item $s=b,r=a,p=u,q=t \tab -\langle ut||ab\rangle \frac{1}{4}$
	\item $s=a,r=b,p=t,q=a \tab +\langle ut||ba\rangle \frac{1}{4}$
\end{itemize}
\tab $\hat{V}$ must involve the orbitals that are different, otherwise mismatch. 

Let us consider first contribution:
\begin{IEEEeqnarray}{rLl}
ut \ t^\dagger u^\dagger \ ba \ a^\dagger b^\dagger&= ut \ t^\dagger u^\dagger \ b\{ a,a\} b^\dagger - \cancel{ ut t^\dagger u^\dagger ba^\dagger ab^\dagger  } \notag \\
&= \wick{
		\c2 u \c1 t \quad \c1{t}^\dagger \c2{u}^\dagger} \{b,b^\dagger \}- \cancel{ ut t^\dagger u^\dagger b^\dagger b  }\cdots |vac\rangle
\end{IEEEeqnarray}
\tab Note that $\{a,a^\dagger\}=1, \{b,b^\dagger\}=1$ and $\hat{a}|vac\rangle = 0$.\\
\tab Other matches without sign change.\\
\tab $\Rightarrow$ The first term has + sign.

For the other terms we can put them in matching order by permuting.\\
\tab Example: 
\begin{IEEEeqnarray}{rLl}
ut \ t^\dagger u^\dagger \ ab \ a^\dagger b^\dagger \langle tu||ba\rangle & = - \wick{\c2 u \c1 t \quad \c1{t}^\dagger \c2{u}^\dagger } \ \wick{\c2 b \c1 a \quad \c1{a}^\dagger \c2{b}^\dagger }\langle tu||ba\rangle \notag \\
&= -\langle tu||ba\rangle \notag \\
&=\langle tu||ab\rangle
\end{IEEEeqnarray}
\tab The same is true for all other contributions.
\begin{IEEEeqnarray}{rLl}
\frac{1}{4}(\langle tu||ab\rangle-\langle tu||ba\rangle -\langle ut||ab\rangle+\langle ut||ba\rangle )& =1 \langle tu||ab\rangle \notag \\
&= \langle tuc \cdots z |\frac{1}{r_{12}} | abc \cdots z\rangle 
\end{IEEEeqnarray}

Let us develop a little further. The device where we indicate which indices are the same.
\begin{itemize}
	\item $\langle pq||rs\rangle \wick{\c2 u \c1 t \quad \c1{p}^\dagger \c2{q}^\dagger } \ \wick{\c2 s \c1 r \quad \c1{a}^\dagger \c2{b}^\dagger }  \rightarrow +\langle tu||ab\rangle$
	\item $\langle pq||rs\rangle \wick{\c2 u \c1 t \quad \c1{p}^\dagger \c2{q}^\dagger } \ \wick{\c2 s \c1 r \quad \c2{a}^\dagger \c1{b}^\dagger }  \rightarrow -\langle tu||ba\rangle$
	\item $\langle pq||rs\rangle \wick{\c2 u \c1 t \quad \c2{p}^\dagger \c1{q}^\dagger } \ \wick{\c2 s \c1 r \quad \c1{a}^\dagger \c2{b}^\dagger }  \rightarrow - \langle ut||ab\rangle$
	\item $\langle pq||rs\rangle \wick{\c2 u \c1 t \quad \c2{p}^\dagger \c1{q}^\dagger } \ \wick{\c2 s \c1 r \quad \c2{a}^\dagger \c1{b}^\dagger }  \rightarrow + \langle ut||ba\rangle$
\end{itemize}
\tab Using lines to indicate the matching indices, we can obtain the indices in: 
\begin{itemize}
	\item [i.] Two-electron integrals.
	\item [ii.] The sign of the expression from the number of intersections of the lines.\\
	$(-)^{\# \text{ crossing} } =sign= $ number of permutations in $V$ operators to undo sign change \\
	Example: \\
	$ \tab \wick{\c2 u \c1 t \quad \c1{p}^\dagger \c2{q}^\dagger } \ \wick{\c2 s \c1 r \quad \c2{a}^\dagger \c1{b}^\dagger } \\
	 = - \wick{\c2 u \c1 t \quad \c1{p}^\dagger \c2{q}^\dagger } \ \wick{\c2 r \c1 s \quad \c1{a}^\dagger \c2{b}^\dagger } \\
	 =-  \wick{\c2 u \c1 t \quad \c1{p}^\dagger \c2{q}^\dagger } \ \wick{\c1 r \c2 s \quad \c2{a}^\dagger \c1{b}^\dagger }  $\\
	( It does not matter how you draw the lines.)
\end{itemize}
We will use this tool to perform contributions, or evaluate matrix-elements in what is to follow. 



\begin{summary}{}{}
The hamiltonian in second quantization needs: 
\begin{center}
	$\hat{H}=\sum_{p,q} h_{pq}\hat{p}^\dagger\hat{q}+\frac{1}{4} \sum_{p,q,r,s} \langle pq||rs\rangle \hat{p}^\dagger \hat{q}^\dagger \hat{s} \hat{r}$
\end{center} 
\tab Determinants can be represented as $|L\rangle=\hat{a}^\dagger\hat{b}^\dagger\cdots\hat{z}^\dagger|vac\rangle $.\\
\tab Then, matrix-elements $\langle K|\hat{H}|L\rangle$ can be evaluated using the anticommutation relations. The results agree with the Slater rules. The hamiltonian applies to system with any number of electrons. \\
\tab There is a very rich field of application of second quantization methods in quantum chemistry. Current developments in wave function based quantum chemistry, and even density matrix based quantum chemistry, invariably are based on this operator approach.\\
\tab Density functional theory is very different and second quantization plays a role there.
\end{summary}

\section{Elementary Applications}
The general sate of a system (closed shell molecule around equilibrium bond length) is often qualitatively well described by single determinant, in which the orbitals are optimized. That single determinant which minimizes the energy $\langle \phi|\hat{H}|\phi\rangle$ is called the Hartree-Fock (HF) determinant and HF energy. This is the topic in the next chapter in $S\&O$ (chapter 3).\\
\tab Let us assume we have solved this problem. 


\begin{itemize}
	\item The HF   state is represented as:  $a^\dagger b^\dagger \cdots c^\dagger |vac\rangle = |HF\rangle$. 
	\item Occupied orbitals $a,b,c,d$ in $|HF\rangle$ are labelled $a,b,c,d,\cdots$ 
	\item The remaining(orthonormal) orbitals , so-called virtual orbitals are labelled $r,s,t,u,\cdots$
	\item  Following $S\& O $ unspecified orbitals have labels $i,j,k,l,\cdots$
\end{itemize}


\subsection{Exitation States}
Excitations out of the $|HF\rangle$ determinant are represented as $\hat{r}^\dagger \hat{a}|HF \rangle$. This means take out orbital $`a'$, then put it in orbital $`r'$(virtual orbital). This is single excitation.\\
\tab NOTE:
\begin{itemize}
	\item   A general single excited state can be expressed as the linear combination.
	\begin{center}
		$\sum_{r,a}\hat{r}^\dagger\hat{a}|HF\rangle C_a^r$
	\end{center}
	\item Double excitations are written as: 
		\begin{center}
		$\sum_{\substack{r<s,\\a<b}}\hat{r}^\dagger \hat{s}^\dagger \hat{b} \hat{a}|HF\rangle C_{ab}^{rs} = \sum_{\substack{r<s,\\a<b}}\hat{r}^\dagger \hat{a} \hat{s}^\dagger \hat{b}|HF\rangle C_{ab}^{rs}$
	\end{center}
	\item  Restricting the summation includes only unique determinants. Improved ground includes combination of excited determinants.
\end{itemize}

Using the proper of second quantization, we can look at states with a different number of electrons, for example:
\begin{itemize}
	\item[1)]  The simplest parameterization of ionized states would be: 
	\begin{center}
		$\sum_a \hat{a}|HF\rangle C_a $
	\end{center} 
	Or in next approximation:
	\begin{center}
		 $(\sum_a\hat{a}|HF\rangle C_a+\sum_{a<b,r}r^\dagger ba|HF\rangle )C_{ab}^{ \ r}$
	\end{center}
	\item[2)] Similarly for states with one more electron (electron attachment):
	\begin{center}
		$\sum_r r^\dagger |HF\rangle C_r $
	\end{center}
	Or in more accuracy:
	\begin{center}
		$\sum_r r^\dagger|HF\rangle C_r + \sum_{r,s,a} r^\dagger s^\dagger a|HF\rangle C_{ra}^{rs}  $
	\end{center}
\end{itemize}



\subsection{Ionization}
This idea includes small number of substitutions or deletions out of the Hartree Fock state to parameterize excitations / ionizations. Then, use the variational principle to obtain matrix-elements $\langle \phi_\lambda | H|\phi_\mu \rangle $. \\
\tab Over limited set of determinants and diagonalize. How to evaluate matrix-elements?\\
\tab $\Rightarrow$ Use techniques of second quantization! \\
\tab For example:
\begin{itemize}
	\item [A.] Ionization energies: 
\begin{IEEEeqnarray}{rLl}
|\psi_\lambda = \sum_a \hat{a}|HF\rangle C_{a,\lambda}
\end{IEEEeqnarray}
	
	\item [B.] Hamiltonian elements: $\langle HF|\hat{b}^\dagger \hat{H} \hat{a}|HF\rangle $  \\
	$a,b$: occupied orbitals in $|HF\rangle$
	\item [C.]Strategy: \\
	More operators $\hat{a},\hat{b}^\dagger$ using anticommutation rules, such that:\\
	 $\hat{a}$ acts on bra: $\langle HF|\hat{a} = (\hat{a}^\dagger |HF\rangle )^\dagger =0 $.\\
	 $\hat{b}$ acts on ket: $\hat{b}^\dagger |HF\rangle =0 $.\\
	 I cannot create occupied orbitals twice $\Rightarrow$ 0. \\
	 This reflects as writing expression in normal order (compare vibrational) \\
	 $\rightarrow$ operators $\hat{b}^\dagger, \hat{r}$ to right.\\
	 $\rightarrow$ operators $\hat{a}, \hat{r}^\dagger$ to left.	 \\
	 $r$ : virtual orbitals: $\hat{r}|HF\rangle =0$, $\langle HF|\hat{r}^\dagger =0$.
\end{itemize}

\subsubsection{Evaluation of One-Electron Integral}
Let us work it out for one-electron elements: 
\begin{IEEEeqnarray}{rLl}
\langle HF|b^\dagger \ \hat{h} \ a |HF\rangle &=\langle HF|b^\dagger \ \sum_{i,j} h_{ij} \ i^\dagger j \ a |HF\rangle  \notag  \\ 
&= \sum_{i,j} h_{ij} \langle HF|b^\dagger i^\dagger j a |HF \rangle \notag \\
&= \sum_{i,j} h_{ij}(-) \langle HF|b^\dagger i^\dagger a j|HF\rangle \notag \\
&= \sum_{i,j} h_{ij} \left[  (-)\delta_{ia} \langle HF|b^\dagger j|HF\rangle +\langle HF|b^\dagger ai^\dagger j|HF\rangle \right] \notag \\
&= \sum_{i,j} h_{ij} \left[  (-)\delta_{ia}\delta_{jb}  +\delta_{ab} \langle HF|i^\dagger  j|HF\rangle \right] \notag \\
&= -h_{ab} +\delta_{ab}\langle HF|\hat{h}|HF\rangle \notag \\
&= -h_{ab} +\delta_{ab}\sum_{c} h_{cc}
\end{IEEEeqnarray}
\tab\tab Note:\\
\tab \tab \tab 1. $i^\dagger a =\{ i^\dagger, a\} - a i^\dagger =\delta_{ia}-ai^\dagger $ \\
\tab \tab  \tab 2. $(\langle HF|a )^\dagger =a^\dagger |HF\rangle =0 $ \\
\tab\tab\tab 3. $a|HF\rangle$ is non-zero term.

Let us also do it in an alternative way, drawing lines to indicate which lines are paired:
\begin{IEEEeqnarray}{rLl}
\sum_{i,j} h_{ij} \left[ \langle HF|  \wick{ \c2{b}^\dagger \ \c1{i}^\dagger \c2 j  \ \c1 a  } |HF \rangle + \langle HF|  \wick{ \c2{b}^\dagger \ \c1{i}^\dagger \c1 j  \ \c2 a  } |HF \rangle \right] =-h_{ab}+\delta_{ab}\sum_c h_{cc}
\end{IEEEeqnarray}
\tab Note that if $\wick{\c1{i}^\dagger \ \c1 j }$ is paired, it means they have to be occupied labels:
\begin{IEEEeqnarray}{rLl}
\sum_{i,j} h_{ij} \langle HF| i^\dagger j| HF\rangle = \sum_c h_{cc}
\end{IEEEeqnarray}
\tab Hence, the expression can reduce to $-h_{ab}+\delta_{ab}\sum_c h_{cc}$ as before.

\subsubsection{Evaluation of Two-Electron Integral}

Let us evaluate $\hat{V}$ contribution: 
\begin{IEEEeqnarray}{rLl}
\langle HF|b^\dagger \ \hat{V} \ a |HF\rangle &= \frac{1}{4} \sum_{i,j,k,l} \langle HF| b^\dagger (i^\dagger  j^\dagger lk) a |HF  \rangle \langle ij||kl\rangle \notag \\
  = &\frac{1}{4} \sum_{i,j,k,l} \langle ij||kl \rangle  [  \langle HF|\wick{ \c2{b}^\dagger \ \c3{i}^\dagger \ \c1{j}^\dagger \ \c2 l \ \c3 k  \ \c1 a |HF\rangle } \notag \\
 +& \langle HF|\wick{ \c2{b}^\dagger \ \c3{i}^\dagger \ \c1{j}^\dagger \ \c2 l \ \c3 k \ \c1 a |HF\rangle }       \notag \\
+& \langle HF| \wick{ \c2{b}^\dagger \ \c1{i}^\dagger \ \c3{j}^\dagger  \ \c2 l \ \c3 k \ \c1 a |HF\rangle }       \notag \\
+& \langle HF| \wick{ \c1{b}^\dagger \ \c2{i}^\dagger \ \c3{j}^\dagger \ \c3 l \ \c2 k \ \c1 a |HF\rangle }      \notag \\
 +& \langle HF|\wick{ \c1{b}^\dagger \ \c2{i}^\dagger \ \c3{j}^\dagger \ \c3 l \ \c2 k \ \c1 a |HF\rangle }        \notag \\
& \begin{aligned}
\\
 =   \frac{1}{4} \sum_c \langle ca||cb \rangle &(-)  \\
 + \frac{1}{4} \sum_c \langle ca||bc \rangle &(+)  \\
 + \frac{1}{4} \sum_c \langle ac||cb \rangle &(+)   \\
+ \frac{1}{4} \sum_c \langle ac||bc \rangle &(-)   \\
 + \ \delta_{ab}\langle HF|\hat{V}| H &F \rangle 
\end{aligned}
\begin{aligned}
&\left.\vphantom{\begin{aligned}
 =   \frac{1}{4} \sum_c \langle ca||cb \rangle &(-)  \\
 + \frac{1}{4} \sum_c \langle ca||bc \rangle &(+)  \\
 + \frac{1}{4} \sum_c \langle ac||cb \rangle &(+)  \\
 + \frac{1}{4} \sum_c \langle ac||bc \rangle &(-)   
  \end{aligned}}\right\rbrace\quad =-\sum_c\langle ac||bc\rangle  
\end{aligned}
\end{IEEEeqnarray}

Hence in total we have obtained: 
\begin{IEEEeqnarray}{rLl}
\langle HF|b^\dagger \hat{H} a |HF \rangle = -h_{ab} - \sum_c \langle ac||bc\rangle +\langle HF|\hat{H}|HF\rangle 
\end{IEEEeqnarray}
\tab Define so-called Fock matrix:
\begin{IEEEeqnarray}{rLl}
f_{ab} = h_{ab}+\sum_{c}\langle ac||bc\rangle
\end{IEEEeqnarray}
\tab Then:
\begin{IEEEeqnarray}{rLl}
\langle HF|\hat{b}^\dagger \hat{H} \hat{a} |HF\rangle = -f_{ab}+E_{HF}\delta_{ab}
\end{IEEEeqnarray}
\tab Diagonalize $F_{ab}$ matrix over occupied orbitals:
\begin{IEEEeqnarray}{rLl}
\sum_b f_{ab} C_{b,\lambda} = \varepsilon_\lambda C_{a,\lambda}
\end{IEEEeqnarray}
\tab Note that ``orbital eigenvalues'' are approximations to ionization energies.
\begin{IEEEeqnarray}{rLl}
E_{IP,\lambda} -E_{HF} &\approx -\varepsilon_\lambda \\
-\varepsilon_\lambda &>0 \\
E_{HF} &< E_{IP}
\end{IEEEeqnarray}
\tab $\varepsilon_\lambda$, eigenvalues of $\hat{f}$ for occupied orbitals are negative (printed in Gaussian).


\subsection{Electron Attachment}
Let us also consider electron attachment: states with one more electron. 
\begin{IEEEeqnarray}{rLl}
\psi_\mu = \sum_r \hat{r}^\dagger |HF\rangle C_{r,\mu}
\end{IEEEeqnarray}
\tab \tab \tab $r$: virtual orbital, not occupied in $|HF\rangle$.

\subsubsection{Evaluation of Matrix-Elements}
Evaluate matrix-elements: $\langle HF|r \ \hat{H} \ s^\dagger|HF\rangle $ \tab $r,s$: virtual orbitals \\
\begin{itemize}
	\item One-electron integral: (use `line' algorithm)
\begin{IEEEeqnarray}{rLl}
& \sum_{i,j} h_{ij} [\langle HF| \wick{\c1 r \ \c1{i}^\dagger \ \c1 j \ \c1{s}^\dagger }|HF\rangle + \langle HF| \wick{\c1 r \ \c2{i}^\dagger \ \c2 j \ \c1{s}^\dagger }|HF\rangle  ] \notag \\
=& h_{rs} +\delta_{rs} \langle HF|\hat{h}|HF\rangle \notag \\
=& h_{rs} +\sum_c h_{cc}\delta_{rs}
\end{IEEEeqnarray}
	\item Two-electron integral: 
\begin{IEEEeqnarray}{rLl}
 \frac{1}{4} \sum_{i,j,k,l}\langle ij||kl\rangle  \bigg [&  \langle HF|\wick{\c1 r \ \c1{i}^\dagger \ \c2{j}^\dagger \ \c2 l \ \c1 k \ \c1{s}^\dagger } |HF  \rangle \Longrightarrow +\langle rc||sc\rangle   \notag \\
 + &  \langle HF|\wick{\c1 r \ \c2{i}^\dagger \ \c1{j}^\dagger \ \c2 l \ \c1 k \ \c1{s}^\dagger } |HF  \rangle \Longrightarrow -\langle cr||sc\rangle \notag \\
+ &\langle HF|\wick{\c1 r \ \c1{i}^\dagger \ \c2{j}^\dagger \ \c1 l \ \c2 k \ \c1{s}^\dagger } |HF  \rangle \Longrightarrow -\langle rc||cs\rangle \notag \\
+ &\langle HF|\wick{\c1 r \ \c2{i}^\dagger \ \c1{j}^\dagger \ \c1 l \ \c2 k \ \c1{s}^\dagger } |HF  \rangle  \Longrightarrow + \langle cr||cs\rangle\notag \\
+ & \langle HF|\wick{\c1 r \ \c2{i}^\dagger \ \c3{j}^\dagger \ \c3 l \ \c2 k \ \c1{s}^\dagger } |HF  \rangle \delta_{rs}\langle \hat{V}\rangle  \bigg ] \notag 
 \end{IEEEeqnarray}
In total the four permutations of $\frac{1}{4} \langle rc||sc\rangle$ add up to $+1\langle rc||sc\rangle$, hence: 
\begin{IEEEeqnarray}{rLl}
\langle HF|r\hat{H} s^\dagger |HF\rangle &= h_{rs} +\sum_c\langle rc||sc\rangle +\delta_{rs}\langle HF|\hat{H}|HF\rangle \notag \\
&\equiv f_{rs}+\delta_{rs}E_{HF}
 \end{IEEEeqnarray}
\end{itemize}

Diagonalize $f_{rs}$ over virtual labels: 
\begin{IEEEeqnarray}{rLl}
f_{rs}C_{s \lambda} = \varepsilon_\lambda C_{r\lambda}
 \end{IEEEeqnarray}
\tab \tab\tab\tab\tab\tab\tab $\Rightarrow \varepsilon_\lambda $: virtual orbital energies. 
\begin{IEEEeqnarray}{rLl}
(E_{(N+1),\lambda}-E_{HF} ) &= \varepsilon_\lambda \\
\varepsilon_\lambda > 0
 \end{IEEEeqnarray}
\tab This would mean one cannot bind an extra electron, and these energies are artifacts of finite basis calculation. Only $\varepsilon_\lambda < 0$ would have solid physical meaning.\\
\tab The ionization energies, eigenvalues of the fock matrix are often reasonable. \\
\tab This is called Koopman's theorem.














\end{document}