\documentclass[a4paper, 12pt]{article}
\usepackage{matnoble-doc-en}
\usepackage{amsmath}
\newcommand\Item[1][]{%
  \ifx\relax#1\relax  \item \else \item[#1] \fi
  \abovedisplayskip=0pt\abovedisplayshortskip=0pt~\vspace*{-\baselineskip}}

\begin{document}

\title{\bf {CHEM400/740: Quantum Mechanics in Chemistry\\ Chapter\#06: Hartree-Fock Algorithm / DFT Theory}} \author{\bf
  \href{http://scienide2.uwaterloo.ca/~nooijen/website_new_20_10_2011/About.html}{Marcel Nooijen}} \date{}
  
\pagestyle{fancy} \fancyhead[L]{\textcolor{PrimaryColor}{CHEM400/740: Quantum Mechanics in Chemistry}} \fancyhead[R]{\textcolor{PrimaryColor}{2021 Winter}}


\maketitle
\tableofcontents

\clearpage


\section{Hartree-Fock: An Algorithm}

Two questions (broadly) are addressed.
\begin{itemize}
	\item [1)] How should we think of Hartree-Fock equations. (interpretation)
	\item [2)] What would it take to write your own Hartree-Fock program? 
\end{itemize}
\tab I will take first about the details involved 2), then discuss 1).

\subsection{Input for a Hartree-Fock Procedure}
We assume we are given the AO integrals over a specific Atomic orbital basis $|\mu\rangle, \mu=1 \cdots N$.
\begin{itemize}
	\item One-electron integrals: $h_{\mu\nu}, S_{\mu\nu}$.
	\item Two-electron integrals: $\langle \mu\nu|\sigma\tau\rangle$ (spatial integrals).
	\item The equations that under the Hartree-Fock approximation were discussed before. Let me summarize. (use spin-orbitals)
	\begin{itemize}
	\item [1)] Anti-symmetrizer: 
			\begin{IEEEeqnarray}{rLl}
	\langle \mu\nu||\sigma \tau\rangle &= 	\langle \mu\nu|\sigma \tau\rangle -	\langle \mu\nu|\tau\sigma \rangle 
	\end{IEEEeqnarray}
		\item [2)] MO's are orthonormal: 
	\begin{IEEEeqnarray}{rLl}
	C^\dagger SC &=1 
	\end{IEEEeqnarray}
		\item [3)] Density matrix: 
	\begin{IEEEeqnarray}{rLl}
	D&= CC^\dagger
	\end{IEEEeqnarray}
		\item [4)] Energy of Hartree-Fock: 
	\begin{IEEEeqnarray}{rLl}
E_{HF} &= \sum_{\mu,\nu} h_{\mu \nu} D_{\nu \mu} + \frac{1}{2}\sum_{\mu,\nu,\sigma,\tau}\langle \mu\nu||\sigma\tau\rangle D_{\sigma \mu}D_{\tau \nu} 
	\end{IEEEeqnarray}
		\item [5)] Fock matrix:
		\begin{IEEEeqnarray}{rLl}
F_{\mu\nu} &= h_{\mu\nu}+\sum_{\nu\tau}\langle \mu\nu||\sigma\tau \rangle D_{\nu\tau} 
	\end{IEEEeqnarray}
		\item [6)] Generalized eigenfunction, where $\varepsilon$ is diagonal (canonical HF orbitals):
		\begin{IEEEeqnarray}{rLl}
FC &= SC\varepsilon
	\end{IEEEeqnarray} 
	\end{itemize}
	
\end{itemize}

This is far from an actual algorithm. Let us consider the pieces first, and then stick them together in an actual program. That is how I often work in practice: From pieces (subroutines) to the full program. 
\subsection{Solving the Generalized Eigenvalue Problem (Construct D)}
We wish to solve for $C, \varepsilon$, having $F,S$ on input.
	\begin{IEEEeqnarray}{rLl}
FC = SC\varepsilon  \\
	C^\dagger SC =1 
	\end{IEEEeqnarray}
\tab To solve this, define $C'$ as $CP$ in equation (or computer program): 
	\begin{IEEEeqnarray}{rLl}
CP &=(S)^{\frac{1}{2}}C \\
CP^\dagger CP &= C^\dagger (S)^{\frac{1}{2}}(S)^{\frac{1}{2}} C \notag \\
&= C^\dagger SC =1 
	\end{IEEEeqnarray}
\tab $(S)^{\frac{1}{2}}C=CP$: orthonormal orbitals. Then: 
	\begin{IEEEeqnarray}{rLl}
FC &= SC\varepsilon \\
F(S)^{-\frac{1}{2}}(S)^{\frac{1}{2}}C &=(S)^{\frac{1}{2}} ((S)^{\frac{1}{2}}C)\varepsilon \\
((S)^{-\frac{1}{2}}F(S)^{-\frac{1}{2}})(CP)&=(CP)\varepsilon \\
F'C'&=C'\varepsilon 
	\end{IEEEeqnarray}
\tab Here is a regular eigenvalue problem we can solve. Next, calculate the $C$ matrix: 
	\begin{IEEEeqnarray}{rLl}
C= (S)^{-\frac{1}{2}}((S)^{\frac{1}{2}}C) =(S)^{-\frac{1}{2}}CP
	\end{IEEEeqnarray}
\tab The first step of calculation is to calculate $(S)^{-\frac{1}{2}}=X$. How to calculate $X$? \\
\tab Diagonalize the $S$ matrix: $S= uSu^\dagger$.\\
\tab \tab \tab \tab \tab \tab \quad $(S)^{-\frac{1}{2}} = u(S)^{-\frac{1}{2}}u^\dagger$\\
\tab Just change the diagonal elements.\\
\tab By diagonalizing $F'$ we get the full set of orbital energies, $NAO$.\\
\tab To get lowest energy state, we usually fill the lowest energy orbitals: occupied MO's. From this calculate density matrix for real orbitals. 
	\begin{IEEEeqnarray}{rLl}
	D_{\mu\nu}&= \sum_{a,occ}C_{\mu a}C_{\nu a}^\dagger \\
	D&= D^\dagger =D^T
	\end{IEEEeqnarray}

\subsection{Construct Fock matrix}
Constructing the Fock-matrix. Treat spin explicitly. 
		\begin{IEEEeqnarray}{rLl}
F_{\mu\nu} &= h_{\mu\nu}+\sum_{\tau\sigma}\langle \mu\tau||\nu\sigma \rangle D_{\sigma\tau} \notag \\
&= h_{\mu\nu}+\sum_{\tau\sigma}\langle \mu\tau|\nu\sigma  \rangle D_{\sigma\tau} -\sum_{\tau\sigma}\langle \mu\tau|\sigma\nu  \rangle D_{\sigma\tau} \notag \\
&= h_{\mu\nu}+J_{\mu\nu}-K_{\mu\nu}
	\end{IEEEeqnarray}
\tab Note that $J{\mu\nu}$ represents Coulomb integral; $K_{\mu\nu}$ represents Exchange integral.\\

The density matrix can be partitioned by two pieces, $D^\alpha$ (spin-up) and $D^\beta$ (spin-down). \\
\tab For closed shells,  $D^\alpha$ is equal to $D^\beta$. However, for open shells we have different $\alpha$ and $\beta$ orbitals: 
		\begin{IEEEeqnarray}{rLl}
D_{\mu\nu}^{(\alpha )} &= \sum_{a,\alpha} C_{\nu(a\alpha )}C_{\mu (a\alpha )}       \\
D_{\mu\nu}^{(\beta )} &=   \sum_{a,\beta} C_{\nu(a\beta )}C_{\mu (a\beta )} 
	\end{IEEEeqnarray}
\tab Then, the Coulomb matrix $J_{\mu\nu}$ in spatial integrals: 
		\begin{IEEEeqnarray}{rLl}
J_{\mu\nu} &= \sum_{\tau\alpha,\sigma\alpha }\langle \mu\tau_\alpha| \nu\sigma_\alpha\rangle D_{\sigma\tau}^{(\alpha )} +  \sum_{\tau\beta,\sigma\beta }\langle \mu\tau_\beta| \nu\sigma_\beta\rangle D_{\sigma\tau}^{(\beta )} \notag \\
&= \sum_{\tau,\sigma}\langle \mu\tau| \nu\sigma\rangle (D^\alpha+D^\beta)_{\sigma\tau } \notag \\
&= \sum_{\tau,\sigma}\langle \mu\tau| \nu\sigma\rangle D^{\text{total}}_{\sigma\tau } \\
J^\alpha &= J^\beta
	\end{IEEEeqnarray}
\tab The Coulomb (dirac) integral only depends on total density matrix. 


Exchange matrix is spin-dependent: 
		\begin{IEEEeqnarray}{rLl}
K_{\mu \nu}^{(\alpha)} &= \sum_{\tau,\sigma}\langle \mu_\alpha \tau_\alpha| \sigma_\alpha\nu_\alpha \rangle D_{\sigma\tau}^{(\alpha )} \\
K_{\mu \nu}^{(\beta)} &= \sum_{\tau,\sigma}\langle \mu_\beta \tau_\beta| \sigma_\beta\nu_\beta \rangle D_{\sigma\tau}^{(\beta )}
	\end{IEEEeqnarray}

For unrestricted open-shell systems (UHF), we can therefore assemble a spin-dependent Fock-matrix:
		\begin{IEEEeqnarray}{rLl}
F_{\mu \nu}^{(\alpha)} &= h_{\mu\nu}+J_{\mu\nu}-K_{\mu\nu}^{(\alpha)} \\
F_{\mu \nu}^{(\beta)} &= h_{\mu\nu}+J_{\mu\nu}-K_{\mu\nu}^{(\beta)}
	\end{IEEEeqnarray}

For closed-shell systems (RHF), a simplification happens: 
	\begin{IEEEeqnarray}{rLl}
D_{\mu\nu}^{(\alpha )}&=D_{\mu\nu}^{(\beta )} \\
F_{\mu\nu}^{(\alpha )}&=F_{\mu\nu}^{(\beta )} \\
D^{\text{total}}_{\sigma\tau }&= D_{\mu\nu}^{(\alpha )}+D_{\mu\nu}^{(\beta )} =2D^{\alpha}  \\
F_{\mu\nu} &= h_{\mu\nu}+\sum_{\tau,\sigma}(2\langle \mu\sigma|\nu\tau\rangle-\langle \mu\sigma|\tau\nu\rangle )
	\end{IEEEeqnarray}

NOTE: \\	
\tab\tab Unrestricted Hartree-Fock (UHF): Separate treatment of $\alpha, \beta$ orbitals.\\
\tab\tab Restricted Hartree-Fock (RHF): Only solve for $\alpha$-orbitals, $\beta$-orbitals are the same.
	
	
\subsection{Calculating Total Energy}
There are two ways to calculate the total energy. First way just uses formulas as written.\\
\tab For open-shell system:
	\begin{IEEEeqnarray}{rLl}
E_{\text{HF}} &= \sum_{\mu \nu} h_{\mu \nu} D_{\mu \nu}+\frac{1}{2} \sum_{\mu,\nu,\sigma,\tau} \langle \mu\nu||\sigma\tau\rangle D_{\nu\tau} D_{\mu\sigma} \notag \\
&= \frac{1}{2}\sum_{\mu\sigma} h_{\mu\sigma}D_{\sigma\mu}+\frac{1}{2}\sum_{\mu,\sigma}(h_{\mu\sigma}+\sum_{\nu,\tau}\langle \mu\nu||\sigma \tau \rangle D_{\nu\tau} )D_{\mu\sigma} \notag \\
&= \frac{1}{2}\sum_{\mu\sigma} h_{\mu\sigma}D_{\sigma\mu} +\frac{1}{2}\sum_{\mu\sigma}f_{\mu\sigma}D_{\mu\sigma} \notag \\
&= \frac{1}{2}\sum_{\mu\sigma} h_{\mu\sigma}D_{\sigma\mu}^{\text{total}}+\frac{1}{2}\sum_{\mu\sigma}f_{\mu\sigma}^{(\alpha )}D_{\mu\sigma}^{(\alpha )}+\frac{1}{2}\sum_{\mu\sigma}f_{\mu\sigma}^{(\beta )}D_{\mu\sigma}^{(\beta )}
	\end{IEEEeqnarray}
\tab For closed-shell system:
	\begin{IEEEeqnarray}{rLl}	
E_{\text{HF}} &= \underbrace{2}_{\alpha \& \beta \text{ spin}} \times \frac{1}{2}( h_{\mu\sigma} + J_{\mu\sigma}) D_{\sigma\mu}
	\end{IEEEeqnarray}	

Different ways to calculate total energy using trace (linear algebra): 
	\begin{IEEEeqnarray}{rLl}	
E_{\text{HF}} &= 2\times \frac{1}{2} \text{Tr}[(h+F)D]
	\end{IEEEeqnarray}
\tab Calculate E right after constructing Fock, not after diagonalization $\Longrightarrow$ new density matrix, $D$\\
\tab $\Longrightarrow$ Your energy of each iteration should be above final converged energy.

Let us describe the full Hartree-Fock program.


	\begin{enumerate}    
\item[1)] Obtain 1-electron and 2-electron integrals. (e.g. store them on a file)
\item[2)] Calculate: \tab\tab\tab \tab$X= (S)^{-\frac{1}{2}}$\\
Initialize:  \tab\tab\tab\tab $ D^{(\text{in})}=0, F^{(\text{in})}=h$\\
Iterate (Loop): \tab\tab\tab $D^{(\text{in})} \tab D_{(\text{in})}^{\alpha},D_{(\text{in})}^{\beta} $\\
Calculate:\tab\tab\tab\tab $F^\alpha,  F^\beta $\\
 Solve the generalized eigenvalue problem: 
     \begin{align*} 
(XF^\alpha X)CP^\alpha &=CP^\alpha \varepsilon_{\alpha} \\
(XF^\beta X)CP^\beta &=CP^\beta \varepsilon_{\beta} \\
C^\alpha &= XCP^\alpha \\
C^\beta &= XCP^\beta \\
D_\alpha^{(\text{out})}&= C^\alpha(C^\alpha )^T \\
D_\beta^{(\text{out})}&= C^\beta(C^\beta )^T 
    \end{align*}
Finally, calculate the energy.


NOTE:
\begin{itemize}
	\item The crux of Hartree-Fock is that we need to iterate.
	\item The Fock matrix depends on input density. Upon diagonalization, we can get new orbitals and new output density.
	\item Convergence is matched if the input and output density are equal.\\
	$\Longrightarrow$ How to check? \\
	$\Longrightarrow$ Monitor convergence: \fbox{Max$_{\mu\nu} |D^{(\sigma)}_{(\text{in})}-D^{(\sigma)}_{(\text{out} )}|<$ threshold (e.g. $10^{-7}$)}. \\
	This means the absolute value of the maximum deviation between input density and output density is below the threshold.  
	\item If convergence, we are done.\\
	If not, then create better guess by using linear combination of density, for example:
		\begin{IEEEeqnarray}{rLl}	
	D_{\text{new}}^{\sigma} = \gamma D_{\text{out}}^{\sigma}+(1-\gamma )D_{\text{in}}^{\sigma} 
		\end{IEEEeqnarray}
	For $\gamma=1$: $D_{\text{new}}=D_{\text{out}}$.\\
	For $\gamma=0.7$: Damped iteration. \\
	In practice, most people/programs use DIIS convergence acceleration. DIIS: Dirac inverse of iterative subspace.(Peter Pulay, 1985).
	
\end{itemize}

\end{enumerate}
\begin{summary}{}{}
 Hartree-Fock Theory: 
 \begin{itemize}
 	\item minimize the energy of a single determinant.
 	\item in a finite AO basis. 	$\Rightarrow$ Roothaan HF calculation.
 	\item iterative procedures to minimize energy: \fbox{self-consistant field method}.\\
 	self-consistant: $F(D^{\text{in}})\rightarrow D^{\text{out}}$, if $D^{\text{in}}=D^{\text{out}}$ for self-consistant solution.\\
 	field: Fock matrix describes $V^{ee}$ in average way (mean field calculations). \\
 	e.g. $J-K$: effective one-electron potential.\\
 	\tab $V^{Ne}+J-K$: full potential. 
 \end{itemize}  
\end{summary}



\subsection{Interpretation of Density Matrix}	
let us assume first er used an orthonormal basis, $|p\rangle$. For orthonormality: 
		\begin{IEEEeqnarray}{rLl}	
\langle \alpha|\beta \rangle &=S_{\alpha\beta} \\
\langle p|q \rangle &=S_{pq} =1 
		\end{IEEEeqnarray}
\tab Here, molecular orbital $D_{pq}$, instead of $D_{\mu\nu}$.
		\begin{IEEEeqnarray}{rLl}	
D_{pq} &=\sum_a C_{pa}C_{qa} \\
\hat{D} &= \sum_{pq} |p\rangle D_{pq}\langle q| 
		\end{IEEEeqnarray}
\tab Note that $\hat{D}$	 acts like a projection operator: 
		\begin{IEEEeqnarray}{rLl}	
\hat{D}^\dagger &= \hat{D} \\
\hat{D}_{pq} &=\hat{D}_{qp} \\
\hat{D}^2 &= \hat{D} \tab \text{(idempotent)}\\
\sum_q D_{pq}D_{qr} &=\sum_q\sum_a C_{pa}C_{qa} \sum_b C_{qb}C_{rb} \notag \\
&= \sum_{a,b}C_{pa} \delta_{ab} C_{rb} \notag \\
&= \sum_a C_{pa}C_{ra} = D_{pr}
		\end{IEEEeqnarray}
\tab Note that use orthonormality in equation(42): $\sum_qC_{qa}C_{qb}=\delta_{ab}$.	

The eigenvalue of a projector are also ``1'' or ``0''. \\
 \tab ``1'' means vector doesn't change. ``0'' means perpendicular. Proof: 
		\begin{IEEEeqnarray}{rLl}	
Du &= u \lambda \\
D\cdot Du =(Du)\lambda &= u \lambda^2 =u \lambda \\
\lambda^2-\lambda &= 0\\
\lambda=1  &\text{ or } \lambda=0
		\end{IEEEeqnarray}

[a]. Occupied orbitals have eigenvalue 1. $a$ for occupied orbitals.
		\begin{IEEEeqnarray}{rLl}	
\sum_q \sum_a C_{pa}( C_{qa}C_{qc}) &= \sum_a C_{pa} \delta_{ac} =C_{pc}
		\end{IEEEeqnarray}	
\tab Since occupied orbitals are all degenerate, we can rotate them, and they remain eigenstates of D, with eigenvalue 1.

[b]. Virtual orbitals have eigenvalue 0.	$r$ for virtual orbitals.
		\begin{IEEEeqnarray}{rLl}	
\sum_q C_{qa}C_{qr} &= 0
		\end{IEEEeqnarray}		
\tab Therefore, $DC_{q\lambda}=0$, for virtual orbitals.	
	
	
\subsubsection{Interpretation}	
$D$ projector on the occupied space. \\
\tab The principle are the same in non-orthonormal basis, but details are more complicated. The symmetric matrix needs to be considered.	 In the AO basis the idempotency of the density matrix is expressed as, $DSD=D$. for: 
		\begin{IEEEeqnarray}{rLl}	
D&= CnC^\dagger \\
DSD &= (CnC^\dagger )S(CnC^\dagger ) \notag \\ 
&= Cn(C^\dagger SC)nC^\dagger  \notag \\
&= Cn1nC^\dagger \tab (n\cdot n=n \text{ diagonal}) \notag \\ 
& =CnC^\dagger \notag \\ 
&=D
		\end{IEEEeqnarray}	
\tab NOTE: The expression,$DSD=D$, analogue of idempotent [projector] in an non-orthonormal basis.\\
\tab Some current Hartree-Fock programs (e.g. Dalton) minimize $E_{\text{HF}}[D]$ under constraints, $DSD = D, Tr(DS)= N_{el}$. 
		\begin{IEEEeqnarray}{rLl}	
Tr(DS) &= Tr(CC^\dagger S) = Tr(C^\dagger S C) = Tr(1)=N_{el}
		\end{IEEEeqnarray}	





\section{Fundamentals of Density Functional Theory}
Density functional theory is from a generational perspective very similar to Hartree-Fock (Kohn-Sham DFT). The energy expression is replaced by $E_{KS}$. 
		\begin{IEEEeqnarray}{rLl}	
E_{KS}= h_{\mu\nu}D_{\mu\nu}+\frac{1}{2}\sum_{\mu\nu\sigma\tau}\langle \mu\nu|\sigma\tau \rangle D_{\sigma\mu}D_{\nu\tau}+E_{xc}[D]
		\end{IEEEeqnarray}	
\tab NOTE:
\begin{itemize}
	\item $\frac{1}{2}\sum_{\mu\nu\sigma\tau}\langle \mu\nu|\sigma\tau \rangle D_{\sigma\mu}D_{\nu\tau}$ is 2-electron Dirac Coulomb term.
	\item$E_{xc}[D]$ is so-called exchange-correlation functional. This gives rise to an additional term to the KS Fock matrix. 
		\begin{IEEEeqnarray}{rLl}	
F = h + J + \frac{\partial E_{xc} }{\partial D_{\mu\nu} }
		\end{IEEEeqnarray}	
	\item Results from DFT depend on `empirical' choice of $E_{ec}$. (e.g. B3Lyp, Bp98, PBE, $\cdots$) \\ Results for geometries and vibrational frequencies are very good. (B3Lyp for molecules)\\ Thermo-chemistry: Not bad (within 3-5 kcal/mol often).
\end{itemize}

\subsection{ Hohenberg–Kohn Theorem}
The Hohenberg–Kohn theorem states that the electronic energy is a functional of the electron density, $\rho(\vec{r})$. Let us first discuss the electron density, for a general wavefunction $\psi(1,2,\cdots,N)$. This quantity naturally derives by considering the expectation value of the $V^{Ne}$ term, so-called external potential.
		\begin{IEEEeqnarray}{rLl}	
\langle V^{Ne} \rangle &= \int \psi^*(1,2,\cdots,N)[\hat{V}^{Ne}(1)+\hat{V}^{Ne}(2)+\cdots+\hat{V}^{Ne}(N) ]\psi(1,2,\cdots,N) d1\cdots dN \\
\hat{V}^{Ne} &= \sum_i \hat{V}^{Ne}(i)
		\end{IEEEeqnarray}

Since electrons are indistinguishable. I can take just $\hat{V}^{Ne}(1)$, and multiply by $N (N_{el})$. 
		\begin{IEEEeqnarray}{rLl}	
\langle V^{Ne} \rangle &= N \int \psi^*(1,2,\cdots,N) \hat{V}(1) \psi(1,2,\cdots,N) d1\cdots dN \notag \\
&= \int d1 \hat{V}(1) \cdot N\int d2\cdots dN \psi^*(1,2,N) \psi(1,2,N)d2 \cdots dN  \notag \\
&= N \int d1 \hat{V}(1) \rho(1) d1
		\end{IEEEeqnarray}

 $\rho(\vec{r})$ is the probability to find an electron at position $\vec{r}$. This is also the distribution of electrons. 
		\begin{IEEEeqnarray}{rLl}	
\rho(\vec{r}) = N\int dr_2dr_3\cdots dr_N \psi^*(r_2,r_3,\cdots, r_N)\psi(r_2,r_3,\cdots, r_N)
		\end{IEEEeqnarray}
\begin{itemize}
	\item For a single Slater determinant:
			\begin{IEEEeqnarray}{rLl}	
\rho(\vec{r}) &=  \sum_a \psi_a^* (\vec{r})\psi_a (\vec{r}) \langle \sigma|\sigma \rangle \notag \\
&= \sum_{a,\alpha} |\psi_a^\alpha (r) |^2+ \sum_{a,\beta}|\psi_a^\beta (r) |^2 \notag \\
&= \rho^\alpha (\vec{r}) + \rho^\beta (\vec{r})
		\end{IEEEeqnarray}
	\item In terms of density matrix: 
\begin{IEEEeqnarray}{rLl}	
\rho(\vec{r}) &=  \sum_{\mu,\nu} \chi^*_\mu(r)\chi_\nu(r)D_{\mu\nu}^{[\sigma]}
		\end{IEEEeqnarray}
\end{itemize}
\tab The density is closely related to the finite basis density matrix. \\
\tab To discuss Hohenberg–Kohn Theorem, consider a family of Hamiltonians that differ in the external potential.
\begin{IEEEeqnarray}{rLl}	
\hat{H}_a &= \hat{T}+V_a+\hat{V}^{ee} \notag \\
&= \hat{T}+V_a+\hat{w}
		\end{IEEEeqnarray}
	NOTE: 
	\begin{itemize}
		\item Kinetic energy, $\hat{T}$ and $\frac{1}{r_{ij}}$ interactions between electrons, $\hat{V}^{ee}$ are always same. 
		\item Often use $\hat{w}$ for $\hat{V}^{ee}$ in DFT literatures.
	\end{itemize}
\tab Then consider two such Hamiltonians $\hat{H}_a, \hat{H}_b$, in which potentials differ (by more than a constant). It is easily shown that such $\hat{H}_a, \hat{H}_b$ have different ground states $\psi_a, \psi_b$. Here is a plausible issue, not a proof: \\
\tab\tab  if $\psi_a=\psi_b$ everywhere, then:
\begin{IEEEeqnarray}{rLl}	
(\hat{H}_a - \hat{H}_b)\psi_a(\vec{r}) &= (E_a-E_b)\psi_a(\vec{r}) \notag \\
&= (V_a(\vec{r})-V_b(\vec{r}))\psi_a(\vec{r}) \notag \\
&= C \psi_a(\vec{r}) 
		\end{IEEEeqnarray}
\begin{IEEEeqnarray}{rLl}	
(V_a(\vec{r})-V_b(\vec{r})-C)\psi_a(\vec{r}) &= 0
		\end{IEEEeqnarray}
\tab \tab Note that $\psi_a(\vec{r})=0$ at all points when $V_a(\vec{r})\neq V_b(\vec{r})+C$.

Let us assume $\psi_a, \psi_b$ are each non-degenerate. Then the Hohenberg–Kohn theorem says that also $\rho_a$ and $\rho_b$	have to be different. Hence: 
\begin{IEEEeqnarray}{rLl}	
\hat{H}_a &= \hat{T}+V_a+\hat{w}; \tab \hat{H}_a\psi_a = E_a\psi_a; \tab \rho_a \\
\hat{H}_b &= \hat{T}+V_b+\hat{w}; \tab \hat{H}_b\psi_b = E_b\psi_b; \tab \rho_b		\end{IEEEeqnarray}		
\tab Conclusion: $\rho_a(\vec{r})\neq \rho_b(\vec{r})$ at least somewhere.	\\
\tab Proof: 	
\begin{IEEEeqnarray}{rLl}	
\langle \psi_b|\hat{H}_a|\psi_b \rangle  &> E_a   \\
\langle \psi_a|\hat{H}_b|\psi_a \rangle &> E_b
	\end{IEEEeqnarray}	
\tab \tab  Used: variational principle, non-dependent ground states. Hence:
\begin{IEEEeqnarray}{rLl}	
(\langle \psi_b|\hat{H}_a|\psi_b \rangle -\overbrace { \langle \psi_a|\hat{H}_a|\psi_a \rangle} ^{E_a}) +  ( \langle \psi_a|\hat{H}_b|\psi_a \rangle -\overbrace { \langle \psi_b|\hat{H}_b|\psi_b \rangle}^{E_b} ) &> 0  \\
\int V_a(1) \rho_b(1) d1 -\int V_a(1)\rho_a(1)d1 + \int V_b(1)\rho_a(1)d1-\int V_b(1)\rho_b(1)d1 &> 0 \\
\int(V_a(1)-V_b(1))(\rho_b(1)-\rho_a(1))d1 &> 0
	\end{IEEEeqnarray}	
 \tab Note that the terms of $\hat{T}+\hat{w}$ cancel (varity in exercise).	
 
 
 	
 \subsection{The Significance of Hohenberg–Kohn Theorem}
\tab Since this is a strict inequity it follows $\rho_a(1)\neq \rho_b(1)$, at least somewhere. Hence, $V_a(1)\leftrightarrows \rho_a(1)$. There is a one-to-one relation between $V_a(1)$ and the corresponding ground state density, $\rho_a(1)$. \\
\tab \tab $\rho_a (1) \rightarrow V_a(1) \rightarrow \hat{H}_a= \hat{T}+V_a+\hat{w} \rightarrow \hat{H}_a\psi_\lambda = E_\lambda \psi_\lambda$. \\
\tab The density has a one-to-one correspondence to $\hat{H}$, and then determine all energies of the system. \\
\tab In quantum chemistry: \\
\tab\tab $\rho(r)$ determine position + charge of nuclei $\rightarrow \hat{H} \rightarrow$ all properties.\\
\tab We can make a mapping: \\
\tab \tab $H_a = \hat{T}+V_a+\hat{w} \rightarrow \psi_a \rightarrow \rho_a $ \\
\tab \tab $E_a= \langle \psi_a|\hat{T}+\hat{w}| \psi_a\rangle +\int V_a(r)\rho_a(r)dr $ \\
\tab \tab $E_{TW}[\rho_a] = \langle \psi_a|\hat{T}+\hat{w}| \psi_a\rangle = E_a -\int V_a(r)\rho_a(r)dr$ \\
\tab Then,
\begin{IEEEeqnarray}{rLl}	
E[\rho] = E_{TW}[\rho] + \int V^{Ne}(r)\rho(r)dr
	\end{IEEEeqnarray}	
\tab This is a functional of the density (a functional associates a number with each function in its domain, here $\rho(r)$). The functional by construction reaches its minimum for $\rho = \rho_g$, and $E = E_0$. \\
\tab The Hohenberg–Kohn principle/proof can be thought of as a construction using external potentials $V_a$. The functional is defined for densities that are ground state densities of Hamiltonian $\hat{H}_a$, with external potential $V_a$. They are called N-representable electron density.\\
\tab  Levy and Lieb made another construction: the constrained-search principle. Define:
\begin{IEEEeqnarray}{rLl}	
E[\rho] = Sub_{\psi \rightarrow \rho} \langle \psi|\hat{H}|\psi \rangle
	\end{IEEEeqnarray}	
\tab Search over all $\psi$ that integrate to $\rho$ and take the energy as the minimum over all such $\psi$. This avoids issues with degeneracies and so-called N-representability.\\
\tab The weakness of both the Levy-Lieb and  Hohenberg–Kohn principle is that they are only existence proofs. We have no explicit way of doing the search. See also the paper I wrote and posted on the website for more discussion. 

\subsection{DFT in Practice}
In Kohn-Sham density functional theory, one minimizes the energy of a single determinant.
\begin{IEEEeqnarray}{rLl}	
E^{KS}=\langle \phi_{KS}|\hat{T}+\hat{V}^{Ne}+\hat{V}^{Hartree}|\phi_{KS} \rangle+ E_{ec}[\rho]
	\end{IEEEeqnarray}
\tab NOTE:
\begin{itemize}
	\item $V^{Hartree}$: Dirac Coulomb term from HF. e.g. $J$ matrix for electron 1, $V(1)=\int \frac{1}{r_{12}}\rho(2)d2$. 
	\item $E_{xc}(\rho)$: Exchange correlation energy.
\end{itemize}
\tab In a finite basis this reduces to:
\begin{IEEEeqnarray}{rLl}	
E^{KS}= h_{\mu\nu}D_{\mu\nu}+\frac{1}{2} \sum_{\mu\nu\sigma\tau} \langle \mu\nu|\sigma\tau \rangle D_{\sigma\mu}D_{\tau\nu}+\int \varepsilon_{ec}(\rho)(1) d1
	\end{IEEEeqnarray}
\tab For example, `local density' approximation: $\varepsilon_{ec}(\rho) = -\frac{2}{3}\rho^{4/3}(1)$. (Dirac's formula for exchange)	\\

 More complicated $E_{ec}(\rho)$ are used. Usually, $E_{ec}$ is found as a functional of $D_{\mu\nu}$, or the molecular orbitals.\\
 \tab A key ingredient is the use of the conventional kinetic energy expression for a single determinant. In practice the same self-consistent field procedure can be used as for a Fock matrix of the form:
 \begin{IEEEeqnarray}{rLl}	
F^{KS}_{\mu\nu}= h_{\mu\nu}+J_{\mu\nu}+\int \frac{\partial E_{xc}}{\partial \rho} \chi^*_{\mu}(1)\chi_{\nu}(1)d1
	\end{IEEEeqnarray}
\tab The last term replace $-K_{\mu\nu}$ from H.F., and is typically evaluated using numerical integration. The total energy upon stationarity is given before.\\
\tab Highly efficient ways of implementing both HF and KS are available. In practice one can evaluate.
 \begin{IEEEeqnarray}{rLl}	
J_{\mu\nu} = \int \chi_\mu^*(1)\chi_\nu(1)d1\int\frac{1}{r_{12}}\rho(r_2)dr_2
	\end{IEEEeqnarray}
	

 DFT, which does not include exact exchange, a fraction of the $K_{\mu\nu}$ matrix, can be implemented very efficiently. In chemistry, most DFT approaches work with exact exchange. They are referred to as hybrid DFT methods.\\
\tab  The above part mainly describes the formal considerations of DFT. 






















\end{document}